% arara: clean: {extensions: ['log','aux','bbl','bcf','blg','run.xml']}
% arara: xelatex: { shell: yes}
% arara: biber
% arara: xelatex: { shell: yes}
% arara: xelatex: { shell: yes}
% arara: clean: {extensions: ['log','aux','bbl','bcf','blg','run.xml']}
\RequirePackage{etex} % Increase the number of tokens up tp 2^15.
\documentclass[11pt,answers,addpoints,a4paper,tittlepage=true,DIV=15,headsepline]{kommaexam}%,cancelspace
\usepackage[T1]{fontenc}
\usepackage{wallpaper}
\usepackage{libertine}
\usepackage[lite]{mtpro2}
\usepackage{amsthm}
\usepackage[libertine,scaled=1.1]{newtxmath}
\usepackage[scaled=.92]{sourcesanspro}
\usepackage[scaled=.78]{beramono}
\usepackage[protrusion=true,expansion=false]{microtype} % IMPORTANT expansion = false for XeTeX engine!!
%\usepackage[spanish, es-sloppy]{babel}%\spanishdatedel
\usepackage{dsfont}
%\usepackage{eucal}
\usepackage{color}
\usepackage{graphicx}
\usepackage{caption}
\usepackage[shortlabels]{enumitem}
\usepackage{ragged2e}
%\usepackage[ddmmyyyy]{datetime}
\usepackage[citestyle=numeric,style=numeric,backend=biber]{biblatex}
\addbibresource{bib.bib}

\pagestyle{headandfoot}
\runningheadrule
\firstpageheader{CM--274 A}{Introducción a la Estadística y las Probabilidades}{2018--3}
\runningheader{CM--274 A}
{Introducción a la Estadística y las Probabilidades}
{Grupo E}
\firstpageheadrule
\firstpagefootrule
\firstpagefooter{Facultad de Ciencias}{Página \thepage\ de \numpages.}{}
%\runningfooter{Facultad de Ciencias}{Página \thepage\ de \numpages.}{Grupo E}
\footer{FC--UNI}
{\iflastpage{Fin del proyecto}{Por favor continúe en la siguiente página\ldots}}
{Página \thepage\ de \numpages.}
\runningfootrule

\renewcommand{\solutiontitle}{\noindent\textbf{Solución}\par\noindent}

\definecolor{SolutionColor}{rgb}{0.2,0.9,1}
\colorsolutionboxes
\definecolor{SolutionBoxColor}{rgb}{0.2,0.9,1}

\hqword{Pregunta}
\hpword{Puntos}
\hsword{Puntaje}

\setkomafont{title}{\huge\sffamily\bfseries}
\setkomafont{subtitle}{\normalfont\large}

\newcommand{\theinstitute}{\\[3\baselineskip]\Large
	Escuela Profesional de Matemática\\
	de la Facultad de Ciencias\\Universidad Nacional de Ingeniería
}
\newcommand{\thethesistype}{}
\newcommand{\thereviewerone}{}
\newcommand{\thereviewertwo}{}
\newcommand{\theeditstart}{}
\newcommand{\theeditend}{}

%% formatting commands for titlepage
\newcommand{\thesistype}[1]{\subtitle{\vskip3em #1 por los alumnos}%
	\renewcommand{\thethesistype}{#1}}
\newcommand{\myinstitute}[1]{\renewcommand{\theinstitute}{#1}}
\newcommand{\reviewerone}[1]{\renewcommand{\thereviewerone}{#1}}
\newcommand{\reviewertwo}[1]{\renewcommand{\thereviewertwo}{#1}}

\newcommand{\editingtime}[2]{%
	\renewcommand{\theeditstart}{#1}%
	\renewcommand{\theeditend}{#2}%
	%% do not show the date
	\date{}
}

\newcommand{\settitle}{%
	\publishers{%
		\large
		\vskip4em
		\begin{tabular}{l l}
		Profesor de teoría: & \thereviewerone\\
		Profesor de Laboratorio: & \thereviewertwo
		\end{tabular}
		\theinstitute\\[4em]
		\vskip4em
		\theeditstart{} -- \theeditend
	}
}

\titlehead{%
	\centering\hspace*{-0.55cm}\includegraphics[height=2.5cm]{1}
	\ThisCenterWallPaper{1}{title-background.pdf}
}
\theoremstyle{definition}
\newtheorem{definition}{Definición}
\newtheorem{theorem}{Teorema}
\newtheorem{corollary}{Corolario}

\usepackage{xpatch}
\makeatletter
\xpatchcmd{\@thm}{\thm@headpunct{.}}{\thm@headpunct{}}{}{}
\makeatother

\author{
	Sarria Palacios Eduardo David\quad\hfill 20141445C\\
	Maquera de la Cruz Stefany Briguitte		\quad\hfill 20162254B\\
	Chung Alvarez Alex Steve	\quad\hfill 20164552C\\
	Aznarán Laos Carlos Alonso	\quad\hfill 20162720C\\
}
\title{Primer proyecto\\
	Introducción a la Estadística y las Probabilidades\\
	CM-274 A}

\thesistype{Informe matemático del grupo E conformado}

%% The reviewers are the professors that grade your thesis
\reviewerone{Ph.D Ángel Enrique Ramírez Gutiérrez.}
\reviewertwo{Lic. José Fernando Zamudio Peves.}

\settitle
%% Please enter the start end end time of your thesis
\editingtime{12 de febrero del 2019}{19 de febrero del 2019}

\begin{document}

\begin{coverpages}
\clearpage\maketitle
\thispagestyle{empty}
\end{coverpages}

%\shadedsolutions

\begin{center}
	\fbox{\fbox{\parbox{5.5in}{\centering
		Datos de los integrantes del Grupo E:
}}}
\end{center}
%\vspace{0.1in}
%\makebox[\textwidth]{Nombre del instructor:\enspace Lic. José Fernando Zamudio Peves.\hfill}

\begin{questions}

	\question%[10]
	Cinco números distintos se distribuyen aleatoriamente entre los jugadores del $1$ al $5$. Cuando dos jugadores comparan sus números, el que tiene el más alto es declarado ganador. Inicialmente, los jugadores $1$ y $2$ comparan sus números; el ganador se compara con el jugador $3$, y así sucesivamente. Sea $X$ el numero de veces que el jugador $1$ es un ganador. Encuentra $\mathds{P}[X = i]$, $i = 0, 1, 2, 3, 4$.
	
	\ifprintanswers
	\begin{figure}[!ht]
		\centering
		\includegraphics[width=7cm]{example-image-a}
		\caption{This is a lovely figure}
		\label{fig:lovely}
	\end{figure}
	\fi
	\begin{solutionorbox}
	Sea $X$ una variable aleatoria cuyos valores de $x$ están en $\left\{1,2,3,4,5\right\}$. Ahora
	\begin{align*}
	f(1) &= \mathds{P}\left(X=1\right)=\frac{}{}, & 	f(2) &= \mathds{P}\left(X=2\right)=\frac{}{}, \\
	f(3) &= \mathds{P}\left(X=3\right)=\frac{}{}, & 	f(4) &= \mathds{P}\left(X=4\right)=\frac{}{}, \\
	f(5) &= \mathds{P}\left(X=5\right)=\frac{}{}, & 			 &
	\end{align*}
	
	Por lo tanto, la distribución de probabilidad de $X$ es
	
	\centering
	\begin{tabular}{c|ccccc}
		$x$ 	& 1 & 2 & 3 & 4 & 5 \\
		\hline
		$f(x)$& 1 & 2 & 3 & 4 & 5
	\end{tabular}

	\end{solutionorbox}
	
	\question%[15]
	Un sistema satelital consta de $n$ componentes y funciona en un día cualquiera si al menos $k$ de los $n$ componentes funcionan ese día. En un día lluvioso, cada uno de los componentes funciona independientemente con probabilidad $p_1$, mientras que en un día seco cada uno funciona independientemente con probabilidad $p_2$. Si la probabilidad de
	que llueva mañana es de $\alpha$, ¿cual es la probabilidad de que el sistema satelital funcione?
	\begin{solutionorlines}
	Hola
	
	\includegraphics[width=7cm]{example-image-b}
	\centering
	\captionof{figure}{This is a lovely figure}
	\label{fig:lovely}
	\justifying
	Hola
	\end{solutionorlines}
	\question%[10]
	Supuesto que el número de accidentes que ocurren en una autopista cada día es una variable aleatoria de \emph{Poisson} con parámetro $\lambda=3$.
	\begin{parts}
		\part Calcular la probabilidad que $3$ o más accidentes ocurran hoy.\label{part:a}
		\part Responder (\ref{part:a}) bajo el supuesto que al menos un accidente ocurrio hoy.
	\end{parts}
	\begin{solutionordottedlines}
	A
	\end{solutionordottedlines}
	\nocolorsolutionboxes
	\question%[10]
	Supuesto que el número de ventas que ocurren en un tiempo específico es una variable aleatoria de Poisson con parámetro $\lambda$. Si cada evento es contado con probabilidad $p$, independiente un evento de otro evento, mostrar que el numero de eventos que son contados es una V.A de Poisson con parámetro $\lambda p$. También, dar un argumento intuitivo de porqué esto debería ser así. Como una aplicación del párrafo anterior, suponer que el número de los distintos depósitos de uranio en un área dada es una variable aleatoria de Poisson con parámetro $\lambda=10$. Si en un periodo de tiempo fijado, cada depósito es descubierto independientemente con probabilidad $\tfrac{1}{50}$, calcular la probabilidad que sea
	\begin{parts}
		\part exactamente $1$.
		\part Al menos $1$
		\part A lo más $1$ depósito son descubiertos durante aquel tiempo.
	\end{parts}
	\begin{solutionorgrid}
	A
	\end{solutionorgrid}

\framedsolutions
	\question%[10]
	Probar que
	\begin{equation*}
	\sum_{i=0}^{n}e^{-\lambda}\frac{\lambda^{i}}{i!}=\frac{1}{n}\int_{\lambda}^{\infty}e^{-x}x^{n}\,\mathrm{d}x
	\end{equation*}
	
	\question Si $X$ es una variable aleatoria geométrica, mostrar analíticamente que
	\begin{equation*}
	\mathds{P}[X=n+k\mid X>n]=\mathds{P}[X=k]
	\end{equation*}
	Dar un argumento verbal usando la interpretación de una variable aleatoria geométrica en cuanto a porqué la ecuación anterior es verdadera.
	
	\begin{solutionorbox}
		A
	\end{solutionorbox}
	
	\unframedsolutions
	
	
	\question Sea $X$ una variable aleatoria binomial negativa con parámetros $r$ y $p$, y sea $Y$ una V.A binomial con parámetros $n$ y $p$. Mostrar que
	\begin{equation*}
	\mathds{P}[X>n]=\mathds{P}[Y<r]
	\end{equation*}
	Sugerencia: La prueba analítica anterior, es equivalente a probar la siguiente identidad
	\begin{equation*}
	\sum_{i=0}^{\infty}\binom{i-1}{i+1}p^{r}{(1-p)}^{i-r}=\sum_{i=0}^{r-1}\binom{n}{i}p^{i}{(1-p)}^{n-i}
	\end{equation*}
	o se podría intentar una prueba que utilice la interpretación probabilística de estas V.A. Es decir, en el ultimo caso, comience considerando una secuencia de ensayos independientes que tienen una probabilidad de éxito común $p$. Luego, trate de expresar los eventos $\{X > n\}$ y $\{Y < n\}$ en términos de los resultados de esta secuencia.
	
	\begin{solutionorbox}
		A
	\end{solutionorbox}
	
	\question Una persona arroja una moneda hasta que aparezca sello por primera vez. Si el sello sale en el $n$-ésimo lanzamiento, la persona gana $2^{n}$ dolares. Si $X$ denota lo ganado por el jugador. Mostrar que $\mathds{E}[X]= +\infty$. Este problema es conocido como la paradoja de San Petersburgo.
	
	\begin{parts}
		\part ¿Estaría dispuesto a pagar \$1 millón para jugar este juego una vez?
		\part ¿Estaría dispuesto a pagar \$1 millón por cada juego si pudiera jugar durante el tiempo que quisiera y solo tendría que conformarse cuando dejó de jugar?
	\end{parts}

	\begin{solutionorbox}
	A
	\end{solutionorbox}

	\question Supongamos que cuando están en vuelo, los motores del avión fallarán con probabilidad $(1-p)$ independientemente de motor a motor. Si un avión necesita la mayoría de sus motores operativos para realizar un vuelo exitoso, ¿Para qué valores de $p$ es preferible un avión de $5$ motores a un avión de $3$ motores?
\end{questions}

%% --------------------
%% |   Bibliography   |
%% --------------------

%% Add entry to the table of contents for the bibliography
\vfill
\nocite{*}
\printbibliography[title={Referencias bibliográficas},heading=bibintoc]

\newpage


\section{Definiciones}
\label{sec:defi}

\subsection{Distribución uniforme}

\begin{definition}
  Una variable aleatoria continua tiene \textit{distribución uniforme} en el intervalo $\left[a,b\right]$ si su función de densidad de probabilidad $f$ es dado por $f(x)=0$ si $x$ no está en $\left[a,b\right]$ y
  \begin{equation*}
    f(x)=\frac{1}{\beta-\alpha}\quad\text{para }\alpha \le x\le \beta.
  \end{equation*}
 Denotamos esta distribución por $U\left(\alpha,\beta\right)$.
\end{definition}

\subsection{La fórmula del cambio de variable}
\label{sec:change}

Con frecuencia no queremos calcular el valor esperado de una variable aleatoria $X$, pero %TODO
de una función $X$, como, por ejemplo, $X^2$. Entonces necesitamos determinar la distribución $Y=X^2$, por ejemplo para calcular la función de distribución $F_Y$ de $Y$ (este es un ejemplo del problema general de cómo las distribuciones bajo las transformaciones -- este tópico es es tema del capitulo 8). Para un ejemplo concreto, suponga un arquitecto quiere maximizar %TODO  Page 94
en el tamaño de los edificios: estos deben ser el mismo ancho y profunidad $X$, pero $X$ es uniformemente distribuido entre $0$ y $10$ metros. ¿Cuál es la distribución del área $X^2$ de un edificio? En particular, ¿será esta distribución (cualquier próxima a la) uniforme? Calculemos $F_Y$, para $0\le a\le 100$:
\begin{equation*}
  F_Y(a)=\mathds{P}\left(X^2\le a\right)=\mathds{P}\left(X\le\sqrt{a}\right)=\frac{\sqrt{a}}{10}.
\end{equation*}
Así, la función de densidad de probabilidad $f_Y$ para el área es, para $0<y<199$ metros cuadrados %TODO usar el paquete siunitx
, dado por
\begin{equation*}
  f_Y(y)=\diff{F_Y(y)}{y}=\diff{\sqrt{y}}{y}=\frac{1}{20\sqrt{y}}.
\end{equation*}
Esto significa que los edificios con áreas pequeñas son %TODO
, porque $f_Y$ explota cerca de 0--vea también la %TODO singularidad
, en el cual %TODO
$f_Y$.

Sorpresivamente, esto no es muy visible en %TODO
, un ejemplo donde debemos crear en nuestros cálculos más que en nuestros ojos. En la figura las ubicaciones de los edificios son generados por el proceso de Poisson, el tema del capítulo 12.

Suponga que un %TODO
tiene que hacer una oferta en el precio de las %TODO
de los edificios. El monto concreto que necesitará será proporcional al área $X^2$ de un edificio. Así, su problema es: ¿cuál es el área esperada de un edificio? Con $f_Y$ de %TODO \eqref{7.1}
él encuentra
\begin{equation*}
  \mathds{E}\left[X^2\right]=\mathds{E}\left[Y\right]=\int_{0}^{199}\! y\cdot\frac{1}{20\sqrt{y}}\dl y=\int_{0}^{100}\frac{\sqrt{y}}{20}\dl y={\left[\frac{1}{20}\frac{2}{3}y^{\tfrac{3}{2}}\right]}_{0}^{100}=33\tfrac{1}{3}\mathrm{m}^2.
\end{equation*}
   %    ODO GRaficar con R la función dada.
La densidad de probabilidad del cuadrado de una variable aleatoria $U\left(0,10\right)$.

Es interesante notar que \emph{realmente} necesitamos hacer este cálculo, porque el valor esperado \emph{no} es simplemente el producto del ancho esperado y la profundidad esperada, que es % \SI{25}{\metre\per\square}

Sin embargo, existe una manera mucho más fácil en el cual el %TODO constructor
puede obtener este resultado. El podría tener el argumento que el valor del \emph{área} es $x^2$ cuando $x$ es el ancho, y que él debería tomar el peso  promedio de \emph{esos} valores, donde el peso de cada ancho $x$ es dado por el valor $f_X(x)$ de la densidad de probabilidad de $X$. Entonces él podría tener calculado
\begin{equation*}
  \mathds{E}\left[X^2\right]=\int_{-\infty}^{\infty}x^2f_X(x)\dl x=\int_{0}^{10}x^2\cdot\frac{1}{10}\dl x={\left[\frac{1}{30}x^3\right]}_{0}^{10}=33\tfrac{1}{3}%TODO.
\end{equation*}
Esto de hecho es un teorema matemático que esto es \emph{siempre} una manera correcta de calcular los valores esperados de funciones de variables aleatorias.

\begin{theorem}[La fórmula del cambio de variable]
  Sea $X$ una variable aleatoria, y sea $g\colon\mathds{R}\rightarrow\mathds{R}$ una función.

  Si $X$ es discreta, tomando los valores $a_1, a_2, \ldots$, entonces
  \begin{equation*}
    \mathds{E}%\left[g\left(\right]
  \end{equation*}
\end{theorem}

\section*{Puntuación}
\begin{center}
	\settabletotalpoints{20}
	\gradetable[h][questions]
\end{center}

\end{document}