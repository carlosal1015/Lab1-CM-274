\section{Definiciones previas}\label{sec:defi}

\subsection{Distribución uniforme}

	\begin{definition}
	Una variable aleatoria continua tiene \textit{distribución uniforme} en el intervalo $\left[a,b\right]$ si su función de densidad de probabilidad $f$ es dado por $f(x)=0$ si $x$ no está en $\left[a,b\right]$ y
	\[
		f(x)=\frac{1}{\beta-\alpha}\quad\text{para }\alpha \le x\le \beta.
	\]
	Denotamos esta distribución por $U\left(\alpha,\beta\right)$.
	\end{definition}

	\begin{figure}[!ht]
		\centering
		\includegraphics[width=0.7\paperwidth]{img5_3}
		\caption{La función de densidad de probabilidad y la función de distribución $U\left(0,\tfrac{1}{3}\right)$.}
	\end{figure}

\subsection{Distribución exponencial}

	Ya encontramos la distribución exponencial en el ejemplo de reactor químico del Capítulo 3. Daremos un argumento de por qué aparece en ese ejemplo. Sea $v$ el caudal volumétrico del efluente, es decir, el volumen que sale del reactor durante un intervalo de tiempo $\left[0,t\right]$ es $vt$ (y un volumen igual entra en el recipiente por el otro extremo). Sea $V$ el volumen del recipiente del reactor. Luego, en total, una fracción $\left(\tfrac{v}{V}\right)\cdot t$ habrá abandonado el recipiente durante $\left[0,t\right]$, cuando $t$ no es demasiado grande. Sea $T$ una variable aleatoria definida como el tiempo de residencia de una partícula en el vaso. Para calcular la distribución de $T$, dividimos el intervalo $\left[0,T\right]$ en $n$ pequeños intervalos de igual longitud $\tfrac{t}{n}$. Suponiendo una mezcla perfecta, de modo que la posición de la partícula se distribuya uniformemente sobre el volumen, la partícula tiene probabilidad $p=\left(\tfrac{v}{V}\right)\cdot\frac{t}{n}$ de haber abandonado el recipiente durante cualquiera de los $n$ intervalos de longitud $\tfrac{t}{n}$. Si asumimos que el comportamiento de la partícula en intervalos de tiempo diferentes de longitud $\tfrac{t}{n}$ son independientes, tenemos, si llamamos ``dejar el recipiente'' un éxito, que $T$ tiene una distribución geométrica con probabilidad de éxito $p$. Eso se sigue que la probabilidad $\mathds{P}\left((T> t)\right)$ de que la partícula aún está en el recipiente en el momento $t$ es, para $n$ grande, bien aproximada por
	\[
		{\left(1-p\right)}^{n}={\left(1-\frac{vt}{Vn}\right)}^{n}.
	\]
	Pero entonces, dejando que $n\to\infty$, obtenemos
	\[
		\mathds{P}\left(T>t\right)=\lim_{n\to\infty}{\left(1-\frac{vt}{V}\cdot\frac{1}{n}\right)}^{n}=e^{-\frac{v}{V}t}.
	\]
	De ello se deduce que la función de distribución de $T$ es igual a $1-e^{-\frac{v}{V}t}$, y diferenciando obtenemos que la función de densidad de probabilidad $f_{T}$ de $T$ es igual a
	\[
		f_{T}=\diff{\left(1-e^{-\frac{v}{V}t}\right)}{t}=\frac{v}{V}e^{-\frac{v}{V}t}\quad\text{para}\quad t\le0.
	\]
	Este es un ejemplo de una distribución exponencial con parámetro $\tfrac{v}{V}$.
	\begin{definition}
		Una variable aleatoria continua tiene una \textit{distribución exponencial} con parámetro $\lambda$ si su función de densidad de probabilidad es dada por $f\left(x\right)=0$ si $x<0$ y
		\[
			f\left(x\right)=\lambda e^{-\lambda x}\quad\text{para }x\le0.
		\]
		Dentamos esta distribución por $\mathrm{Exp}\left(\lambda\right)$.
	\end{definition}
	La función de distribución $F$ de una distribución $\mathrm{Exp}\left(\lambda\right)$ es dada por
	\[
		F\left(a\right)=1-e^{-\lambda a}\quad\text{para}\quad a\le 0.
	\]
	En la Figura~\ref{fig:5.4} mostramos la función de densidad de probabilidad y la función de distribución de la distribución de $\mathrm{Exp}\left(\num{0.25}\right)$.

	\begin{figure}[!ht]
		\centering
		\includegraphics[width=0.7\paperwidth]{img5_4}
		\caption{La función de densidad de probabilidad y la función de distribución de la distribución $\mathrm{Exp}\left(\num{0.25}\right)$.}
		\label{fig:5.4}
	\end{figure}

	Ya que obtuvimos la distribución exponencial directamente de la distribución geométrica, no debería sorprender que la distribución exponencial también satisfaga la propiedad \textit{sin memoria}, es decir, si $X$ tiene una distribución exponencial, entonces para todos $s,t>0$s,
	\[
		\mathds{P}\left(X>s+t\mid X>s\right)=\mathds{P}\left(X>t\right).
	\]
	En realidad, esto se sigue directamente de
	\[
		\mathds{P}\left(X>s+t\mid X>s\right)=\frac{\mathds{P}\left(X>s+t\right)}{\mathds{P}\left(X>s\right)}=\frac{e^{-\lambda\left(s+t\right)}}{e^{-\lambda s}}
		=\mathds{P}\left(X>t\right).
	\]

\subsection{La fórmula del cambio de variable}\label{subsec:change}

	Con frecuencia no queremos calcular el valor esperado de una variable aleatoria $X$, sino más bien una función de $X$, como, por ejemplo, $X^2$. Entonces necesitamos determinar la distribución $Y=X^2$, por ejemplo para calcular la función de distribución $F_Y$ de $Y$ (este es un ejemplo del problema general de cómo las distribuciones bajo las transformaciones--este tópico es es tema del capitulo 8). Para un ejemplo concreto, suponga un arquitecto quiere una variedad máxima en el tamaño de los edificios: estos deben ser el mismo ancho y profundidad $X$, pero $X$ es uniformemente distribuido entre $0$ y $10$ metros. ¿Cuál es la distribución del área $X^2$ de un edificio? En particular, ¿será esta distribución (cualquier próxima a la) uniforme? Calculemos $F_Y$, para $0\le a\le 100$:
	\[
		F_Y(a)=\mathds{P}\left(X^2\le a\right)=\mathds{P}\left(X\le\sqrt{a}\right)=\frac{\sqrt{a}}{10}.
	\]
	Así, la función de densidad de probabilidad $f_Y$ para el área es, para $0<y<100$ metros cuadrados, dado por
	\begin{equation}\label{eq:7.1}
		f_Y(y)=\diff{F_Y(y)}{y}=\diff{\sqrt{y}}{y}=\frac{1}{20\sqrt{y}}.
	\end{equation}
	Esto significa que los edificios con áreas pequeñas son muy sobrerrepresentado, porque $f_Y$ explota cerca de 0--vea también la Figura~\ref{fig:7.3}, en el cual trazamos $f_Y$.

	Sorpresivamente, esto no es muy visible en la Figura~\ref{fig:7.4}, un ejemplo donde debemos crear en nuestros cálculos más que en nuestros ojos. En la figura las ubicaciones de los edificios son generados por el proceso de Poisson, el tema del capítulo 12.

	Suponga que un contratista tiene que hacer una oferta en el precio de las cimientos de los edificios. El monto concreto que necesitará será proporcional al área $X^2$ de un edificio. Así, su problema es: ¿cuál es el área esperada de un edificio? Con $f_Y$ de \ref{eq:7.1} él encuentra
	\[
		\mathds{E}\left[X^2\right]=\mathds{E}\left[Y\right]
		=\int_{0}^{199}\! y\cdot\frac{1}{20\sqrt{y}}\dl y=\int_{0}^{100}\frac{\sqrt{y}}{20}\dl y
		={\left[\frac{1}{20}\frac{2}{3}y^{\tfrac{3}{2}}\right]}_{0}^{100}=33\tfrac{1}{3}\mathrm{m}^2.
	\]

	\begin{figure}[!ht]
		\centering
		\includegraphics[width=0.7\paperwidth]{img7_3}
		\caption{La densidad de probabilidad del cuadrado de una variable aleatoria $U\left(0,10\right)$.}\label{fig:7.3}
	\end{figure}

	Es interesante notar que \emph{realmente} necesitamos hacer este cálculo, porque el valor esperado \emph{no} es simplemente el producto del ancho esperado y la profundidad esperada, que es \SI{25}{\metre\per\square}. Sin embargo, existe una manera mucho más fácil en el cual el contratista puede obtener este resultado. El podría tener el argumento que el valor del \emph{área} es $x^2$ cuando $x$ es el ancho, y que él debería tomar el peso  promedio de \emph{esos} valores, donde el peso de cada ancho $x$ es dado por el valor $f_X(x)$ de la densidad de probabilidad de $X$. Entonces él podría tener calculado
	\[
		\mathds{E}\left[X^2\right]=\int_{-\infty}^{\infty}x^2f_X(x)\dl x=\int_{0}^{10}x^2\cdot\frac{1}{10}\dl x={\left[\frac{1}{30}x^3\right]}_{0}^{10}=33\tfrac{1}{3}\mathrm{m}^{2}.
	\]
	Esto de hecho es un teorema matemático que esto es \emph{siempre} una manera correcta de calcular los valores esperados de funciones de variables aleatorias.

	\begin{figure}[!ht]
		\centering
		\includegraphics[width=0.7\paperwidth]{img7_4}
		\caption{Arriba: anchos de los edificios entre $0$ y $10$ metros. Abajo: los correspondientes edificios en un área de \num{100}$\times$\SI{300}{\metre}.}\label{fig:7.4}
	\end{figure}

	\begin{theorem}[La fórmula del cambio de variable]
		Sea $X$ una variable aleatoria, y sea $g\colon\mathds{R}\rightarrow\mathds{R}$ una función.

		Si $X$ es discreta, tomando los valores $a_1, a_2, \ldots$, entonces
		\[
			\mathds{E}\left[g\left(X\right)\right]=\sum_{i}g\left(a_{i}\right)\mathds{P}\left(X=a_{i}\right).
		\]

		Si $X$ es continua, con función de densidad $f$, entonces
		\[
			\mathds{E}\left[g\left(X\right)\right]=\int_{-\infty}^{\infty}g\left(x\right)f\left(x\right)\dl x.
		\]
	\end{theorem}