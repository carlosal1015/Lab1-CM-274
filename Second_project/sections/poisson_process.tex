\section{El proceso de Poisson}
\label{sec:12}

En varios fenómenos aleatorios encontramos, no solo uno o dos variables aleatorias que juegan un rol, sino una colección completa. En este caso frecuentemente habla de un \textit{proceso} aleatorio. El proceso de Poisson es un tipo simple de proceso aleatorio, que modela la ocurrencia de puntos aleatorios en el espacio o tiempo. Existen números formas en el que proceso de puntos aleatorios aumenta: algunos ejemplos son presentados en la primera sección. El proceso de Poisson describe en cierto sentido la \textit{forma más aleatoria} para distribuir puntos en el espacio o tiempo. Este es hecho más preciso con las nociones de homogeneidad e independencia.

\subsection{Los puntos aleatorios}

Ejemplos típicos de la ocurrencia de puntos aleatorios son: tiempos de llegadas de mensajes de correo electrónico en un servidor, las veces en que los asteroides golpean la tierra, tiempos de llegada de materiales radiactivos en un contador Geiger\footnote{Un contador Geiger es un instrumento que permite medir la radiactividad de un objeto o lugar.}, veces en el que su computadora falla y los tiempos en que fallan los componentes electrónicos y los tiempos de llegada de personas a una bomba de agua en un oasis.

Ejemplos de la ocurrencia de los puntos aleatorios son: las ubicaciones de los impactos de asteroides con la tierra (dimensión $2$), las ubicaciones de las imperfecciones un material (dimensión $3$), y las ubicaciones de los árboles en un bosque. (dimensión $2$).

Algunos de estos fenómenos son mejor modelados por el proceso de Poisson que por otros. De forma aproximada, uno podría decir que el proceso de Poisson se aplica con frecuentemente a situaciones donde existe una población muy grande, y cada miembro de este tiene una pequeña probabilidad de producir un punto del proceso. Esto es, por ejemplo, bien satisfecho en el ejemplo del contador Geiger donde, en una descomunal colección de átomos, solo unos pocos emitirán una partícula radiactiva. Una propiedad del proceso de Poisson--como veremos en breve--es que los puntos pueden estar arbitrariamente juntos. Por lo tanto, las ubicaciones de los árboles no están tan bien modeladas por el proceso de Poisson.

\subsection{Echando un vistazo más de cerca las llegadas aleatorias}

Un ejemplo bien conocido que es usualmente modelado por el proceso de Poisson es que las llamadas que llegan a una central telefónica--la central está conectada a un gran número de personas que hacen llamadas telefónicas de vez en cuando. Este será nuestro ejemplo principal en esta sección.

Las llamadas telefónicas llegan en tiempos aleatorios $X_{1}, X_{2}, \ldots$ a la central telefónica durante el tiempo de tiempo $\left[0,t\right]$.

% TODO Agregar imagen

Los dos supuestos básicos que hacemos en estas llamadas aleatorias son
\begin{description}
	\item[Homogeneidad] La tasa $\lambda$ a la que se producen las llamadas es constante a través del tiempo: en un subintervalo de longitud $u$ la esperanza del número de llamadas telefónicas es $\lambda u$.
	\item[Independencia] El número de llegadas en intervalos de tiempo disjuntos son variables aleatorias independientes.
\end{description}

La homogeneidad es también llamado \textit{}