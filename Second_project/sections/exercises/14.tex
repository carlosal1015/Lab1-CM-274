\question
	En este ejercicio nos fijamos en un ejemplo simple de variables aleatorias $X_n$ que tiene la propiedad que sus distribuciones convergen a las distribuciones de una variable aleatorias $X$ cuando $n\to\infty$. Sea para $n=1,2,\ldots$ las variables aleatorias definidas por
	\begin{equation*}
	P\left(X_n=0\right)=1-\frac{1}{n}\quad\text{y}\quad\mathds{P}\left(X_n=7n\right)=\frac{1}{n}.
	\end{equation*}
	\begin{parts}
		\part Sea $X$ una variable aleatorias que es igual a $0$ con probabilidad $1$. Muestre que para todo $a$, las funciones de masa de probabilidad $p_{X_m}(a)$ de $X_n$ converge a la función de probabilidad de masa $p_X(a)$ de $X$ cuando $n\to\infty$. Note que $\mathds{E}\left[X\right]=0$.
		\part Muestre que sin embargo $\mathds{E}\left[X_n\right]=7$ para todo $n$.
	\end{parts}

	\begin{solutionordottedlines}
		
	\end{solutionordottedlines}