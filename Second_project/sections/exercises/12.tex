\question
	Consideremos el proceso de Poisson unidimensional. Suponga que para algunos $\alpha>0$ se da que hay exactamente dos puntos $\left[0,a\right]$, o en otras palabras: $N_\alpha=2$. El objetivo de este ejercicio es determinar la distribución conjunta de $X_1$ y $X_2$, las ubicaciones de los dos puntos, sujeto a $N_\alpha=2$.

	\begin{parts}
		\part Pruebe que para $0<s<t<\alpha$\label{12.12a}
		\begin{equation*}
			\mathds{P}\left(X_1\le X_2\le t, N_\alpha=2\right)=\mathds{P}\left(X_2\le t, N_\alpha=2\right)-\mathds{P}\left(X_1>s,X_2\le t, N_\alpha=2\right).
		\end{equation*}
		\part Deduzca de~\ref{12.12a} que\label{12.12b}
		\begin{equation*}
			\mathds{P}\left(X_1\le s,X_2\le t,N_\alpha=2\right)=e^{-\lambda\alpha}\left(\frac{\lambda^2t^2}{2!}-\frac{\lambda^2{\left(t-s\right)}^2}{2!}\right).
		\end{equation*}
		\part Deduzca de~\ref{12.12b} que $0<s<t<\alpha$
		\begin{equation*}
			\mathds{P}\left(X_1\le s, X_2\le t\mid N_\alpha=2\right)=\frac{t^2-{\left(t-s\right)}^2}{\alpha^2}.
		\end{equation*}
	\end{parts}

	\begin{solutionordottedlines}
		
	\end{solutionordottedlines}