\question

	Sea $X$ una variable aleatorias con una distribución $\mathrm{Pois}\left(\mu\right)$. Muestre lo siguiente. Si $\mu<1$, entonces las probabilidades $\mathds{P}\left(X=k\right)$ son primero crecientes, luego decreciente (confer Figura~\ref{fig:12_1}). ¿Qué ocurre si $\mu=1$?

	\begin{figure}[ht!]
		\centering
		\includegraphics[width=0.70\paperwidth]{img12_1}
		\caption{Las funciones de probabilidad de masa de las distribuciones de $\mathrm{Pois}\left(0.9\right)$ y $\mathrm{Pois}\left(5\right)$.}
		\label{fig:12_1}
	\end{figure}

	\begin{solutionordottedlines}
		Definimos las probabilidades
		\[
			p_{k}\stackrel{\text{def}}{=}\mathds{P}\left(X=k\right)=\frac{\mu^{k}e^{-\mu}}{k!}\quad\text{para } k=0,1,\ldots
		\]
		Ahora, calculemos el siguiente cociente
		\[
			\frac{p_{k+1}}{p_{k}}
			=\frac{\mathds{P}\left(X=k+1\right)}{\mathds{P}\left(X=k\right)}
			=\frac{\frac{\mu^{k+1}e^{-\mu}}{\left(k+1\right)!}}{\frac{\mu^{k}e^{-\mu}}{k!}}
			=\frac{\frac{\cancel{\mu^{k}}\cdot\mu\cdot \cancel{e^{-\mu}}}{\left(k+1\right)\cdot\cancel{ k!}}}{\frac{\cancel{\mu^{k}}\cdot\cancel{e^{-\mu}}}{\cancel{k!}}}
			=\frac{\mu}{k+1}.
		\]
		Además, si $p_{k}<p_{k+1}$ o $p_{k}>p_{k+1}$, entonces $\frac{p_{k+1}}{p_{k}}>1$ o $\frac{p_{k+1}}{p_{k}}<1$. De esto se sigue que
		\begin{itemize}
			\item Para $\mu<1$, la probabilidad $\mathds{P}\left(X=k\right)$ es decreciente, y
			\item para $\mu>1$, la probabilidad es creciente siempre que $\tfrac{\mu}{k+1}>1$ y decreciente desde el momento en el que esta fracción se vuelve menor que uno.
		\end{itemize}
		Por último, si $\mu={1}$, entonces $p_{0}=p_{1}>p_{2}>p_{3}>\cdots$.
	\end{solutionordottedlines}