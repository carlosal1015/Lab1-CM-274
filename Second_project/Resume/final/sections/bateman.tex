\section{Una nota}
\label{sec:note}

Sea $\lambda\dl t$ la probabilidad que una partícula $\alpha$ golpee la pantalla en un intervalo de tiempo $\dl t$. Si los intervalos de tiempo bajo la consideración son pequeños comparados con el período de tiempo de la sustancia radiactiva, podemos asumir que $\lambda$ es independiente de $t$. Ahora, sea $W_n(t)$ la probabilidad que $n$ partículas $\alpha$ golpeen la pantalla en un intervalo de tiempo $t$, entonces la probabilidad que $(n+1)$ partículas golpeen la pantalla en un intervalo $t+\dl t$ es la suma de dos probabilidades. En primer lugar, las $n+1$ partículas $\alpha$ pueden golpear la pantalla en el intervalo $t$ y ninguno en el intervalo $\dl t$. La probabilidad de que esto pueda ocurrir es $(1-\lambda\dl t)W_{n+1}(t)$. En segundo lugar, $n$ partículas $\alpha$ podrían golpear la pantalla en el intervalo $t$ y otra partícula $\alpha$ en el intervalo $\dl t$; la probabilidad de que esto ocurra es $\lambda\dl t W_n(t)$. Por lo tanto
\begin{equation*}
	W_{n+1}\left(t+\dl t\right)=\left(1-\lambda\dl t\right)W_{n+1}(t)+\lambda\dl t W_n\left(t\right).
\end{equation*}
Procediendo al límite, tenemos
\begin{equation*}
	\diff{W_{n+1}}{t}=\lambda\left(W_n-W_{n+1}\right).
\end{equation*}
Poniendo $n=0,1,2,\ldots$ en una sucesión tenemos el sistema de ecuaciones:
\begin{align*}
	\diff{W_0}{t}&=-\lambda W_0,\\
	\diff{W_1}{t}&=-\lambda\left(W_0-W_1\right),\\
	\diff{W_2}{t}&=-\lambda\left(W_1-W_2\right),\\
	\vdots &=\vdots\\	% TODO
\end{align*}
que son exactamente de la misma forma que las que se producen en la teoría de transformaciones radiactivas, excepto que los períodos de tiempo de las transformaciones deberían suponerse que son todos iguales.

Las ecuaciones podrían ser resueltos multiplicando cada uno de ellos por $e^{\lambda t}$ e integrando. Dado que $W_0(0)=1$ y $W_n(0)=0$, tenemos una sucesión:
\begin{align*}
	&& W_0&= e^{-\lambda t},\\
	\diff{W_1e^{\lambda t}}{t}&=\lambda, & W_1&= e^{-\lambda t},\\
	\diff{W_2e^{\lambda t}}{t}&=\lambda^2t, & W_2&=\frac{{\left(\lambda t\right)}^2}{2!}e^{-\lambda t},\\
\end{align*}
y así sucesivamente. Finalmente, tenemos
\begin{equation*}
	W_n=\frac{{\left(\lambda t\right)}^n}{n!}e^{\lambda t}.
\end{equation*}
El \emph{promedio} del número de  partículas $\alpha$ que golpean la pantalla en el intervalo $t$ es $\lambda t$. Poniendo esto igual a $x$, vemos que la probabilidad de que $n$ partículas $\alpha$ golpeen la pantalla en este intervalo es
\begin{equation*}
	W_n=\frac{x^n}{n!}e^{-x}.
\end{equation*}

El caso particular en el que $n=0$ se conoce desde hace tiempo. % TODO Chequear el libro Whitworth Choice and Chance 4th ed prop 51.

Si usamos la analogía de arriba con la transformación radiactiva, el teorema simplemente nos dice que la cantidad de sustancia primaria restante después de un intervalo de tiempo $t$ es $e^{-\lambda t}$ si una cantidad unitaria estaba presente en el comienzo.

El número \emph{probable} de partículas $\alpha$ que golpean la pantalla en un intervalo dado es
\begin{equation*}
	p=\sum_{n=1}^{\infty}nW_n=xe^{-x}\sum_{n=1}^{\infty}\frac{x^{n-1}}{\left(n-1\right)!}=x.
\end{equation*}
El número \emph{más probable} es obtenido por la búsqueda del valor máximo de $W_n$.

Ya que $\frac{W_n}{W_{n-1}}=\frac{x}{n}$, este radio será mayor que $1$ siempre que $n<x$. Por lo tanto, si $n\lessgtr x$, %TODO el simbolo
\begin{equation*}
	W_n\lessgtr W_{n-1}%TODO el sombolo
\end{equation*}
si $n=x$, $W_n=W_{n-1}$. El valor más probable de $n$ es por lo tanto el entero siguiente mayor que $x$; si, sin embargo, $x$ es un entero, los números $x-1$ y $x$ son igualmente probable, y más probable que todos los otros.

El valor de $\lambda$ que es calculado por el conteo del número total de partículas $\alpha$ que chocan la pantalla en un intervalo largo de tiempo $T$, no será generalmente el valor verdadero de $\lambda$. La desviación media desde el valor de verdad de $\lambda$ es calculado por la búsqueda de la desviación media del número total $N$ de partículas $\alpha$ observadas en el tiempo $T$ desde el número verdadero promedio $\lambda T$. Esta desviación media $D$ (error promedio) es, de acuerdo a la definición de Bessel y Gau\ss, la raíz cuadrada del valor probable del cuadrado de la diferencia $N-\lambda T$, y así es obtenido por las series % TODO Footnote de Bessel y Gauss.
\begin{align*}
	D^2
	&=\sum_{n=0}^{\infty}\left(N-\lambda T\right)^2\frac{{\left(\lambda T\right)}^N}{N!}e^{-\lambda t} &\\
	&=e^{-\lambda t}\sum_{N=0}^{\infty}\left[\frac{{\left(\lambda T\right)}^N}{\left(N-2\right)!}+\frac{{\left(\lambda T\right)}^{N+1}}{\left(N-1\right)!}+\frac{{\left(\lambda T\right)}^{N+2}}{\left(N\right)!}\right]=\lambda T.
\end{align*}
Así, $D=\sqrt{\lambda T}$, y la desviación media desde el valor de $\lambda$ es en consecuencia
\begin{equation*}
  \frac{D}{T}=\sqrt{\frac{\lambda}{T}};
\end{equation*}
esto es, varía inversamente como la raíz cuadrada de la longitud del intervalo de tiempo. Este resultado es de la misma forma que el clásico utilizado por E. v. Schweidler en el artículo  mencionado anteriormente.

El valor probable de $|N-\lambda T|$ (error promedio) es mucho más difícil de calcular.

% TODO citar Schwediler

%%% Local Variables:
%%% mode: latex
%%% TeX-master: "../main"
%%% End: