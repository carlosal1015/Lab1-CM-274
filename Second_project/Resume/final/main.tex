% arara: clean: {extensions: ['log','aux','bbl','bcf','blg','run.xml','fls','fdb_latexmk','synctex.gz','toc']}
% arara: xelatex: {shell: yes, synctex: yes}
% arara: xelatex: {shell: yes, synctex: yes}
% arara: clean: {extensions: ['log','aux','bbl','bcf','blg','run.xml','fls','fdb_latexmk','synctex.gz','toc']}
\documentclass[
	10pt,
	twoside=false,
	twocolumn=false,
	BCOR=12mm,
	DIV=calc
]{scrartcl} %,headlines=2.1,footlines=2.1
%\usepackage[paper=a4,pagesize]{typearea} %headinclude=false,footinclude=false
\usepackage[onehalfspacing]{setspace}
\AfterTOCHead{\singlespacing}
\KOMAoptions{DIV=last}

\usepackage[T1]{fontenc}
\usepackage[spanish,es-sloppy]{babel}
	\spanishdatedel
\usepackage{libertine}
\usepackage[lite]{mtpro2}
\usepackage{amsthm}
\usepackage[libertine,scaled=1.1]{newtxmath}
\usepackage[scaled=.92]{sourcesanspro}
\usepackage[scaled=.78]{beramono}
\usepackage[ISO]{diffcoeff}
\usepackage{dsfont}
\usepackage{siunitx}
	\sisetup{binary-units=true,per-mode=symbol}
\usepackage[citestyle=numeric,style=numeric,backend=biber]{biblatex}
\addbibresource{./bibliography/bib.bib}
\addbibresource{bib.bib}
\theoremstyle{definition}
\newtheorem{definition}[section]{Definición}
\newtheorem{theorem}[section]{Teorema}

\setlength{\marginparwidth}{1.5\marginparwidth}

\begin{document}

\tableofcontents

\begin{titlepage}
	\extratitle{\textbf{\LARGE Traducción del artículo\\[\baselineskip] \textit{Las variaciones de la probabilidad en la\linebreak distribución de partículas $\alpha$}}}
	\title{Sobre la distribución de probabilidad de las partículas $\alpha$}
	\author{Harry Bateman}
	\begin{spacing}{1}
		\maketitle
	\end{spacing}
	\thispagestyle{empty}
\end{titlepage}

\documentclass[10pt,twoside=false,twocolumn=false,BCOR=12mm,DIV=calc]{scrartcl} %,headlines=2.1,footlines=2.1
%\usepackage[paper=a4,pagesize]{typearea} %headinclude=false,footinclude=false
\usepackage[onehalfspacing]{setspace}
\AfterTOCHead{\singlespacing}
\KOMAoptions{DIV=last}
\usepackage[T1]{fontenc}
\usepackage[spanish,es-sloppy]{babel}
\spanishdatedel
\usepackage{libertine}
\usepackage[lite]{mtpro2}
\usepackage{amsthm}
\usepackage[libertine,scaled=1.1]{newtxmath}
\usepackage[ISO]{diffcoeff}
% \providecommand{\dt}{\mathrm{d}t}
\usepackage{dsfont}
\usepackage{siunitx}

\theoremstyle{definition}
\newtheorem{definition}[section]{Definición}
\newtheorem{theorem}[section]{Teorema}

\setlength{\marginparwidth}{1.5\marginparwidth}

\begin{document}

\begin{titlepage}
  \extratitle{\textbf{\LARGE Traducción del artículo\\[\baselineskip] \textit{Las variaciones de la probabilidad en la distribución de partículas $\alpha$}}}
  \title{Sobre la distribución de probabilidad de las partículas $\alpha$}
  \author{H. Bateman}
  \begin{spacing}{1}
    \maketitle
  \end{spacing}
  \thispagestyle{empty}
\end{titlepage}

\tableofcontents
\newpage
\section{Una nota}
\label{sec:note}


Sea $\lambda t$ la probabilidad que una partícula $\alpha$ golpee la pantalla en un intervalo de tiempo $\dl t$. Si los intervalos de tiempo bajo la consideración son pequeños comparados con el período de tiempo de la sustancia radiactiva, podemos asumir que $\lambda$ de $t$. Ahora, sea $W_n(t)$ la probabilidad que $n$ partículas $\alpha$ golpeen la pantalla en un intervalo de tiempo $t$, entonces la probabilidad que $(n+1)$ partículas golpeen la pantalla en un intervalo $t+\mathrm{d}t$ es la suma de las dos probabilidades. En primer lugar, las $n+1$ partículas $\alpha$ pueden golpear la pantalla en el intervalo $t$ y ninguno en el intervalo $\mathrm{d}t$. La probabilidad de que esto pueda ocurrir es $(1-\lambda t\mathrm{d}t)W_{n+1}(t)$. En segundo lugar, $n$ partículas $\alpha$ podrían golpear la pantalla en el intervalo $t$ y una partícula en el intervalo $\mathrm{d}t$; la probabilidad de que esto ocurra es $\lambda\mathrm{d}t W_n(t)$. Por lo tanto
\begin{equation*}
  W_{n+1}\left(t+\mathrm{d}t\right)=\left(1-\lambda\mathrm{d}t\right)W_{n+1}(t)+\lambda\mathrm{d}t W_n\left(t\right).
\end{equation*}
Procediendo al límite, tenemos
\begin{equation*}
  \frac{\mathrm{d}W_{n+1}}{\mathrm{d}t}=\lambda\left(W_n-W_{n+1}\right).
\end{equation*}
Poniendo $n=0,1,2,\ldots$ en una sucecsión tenemos el sistema de ecuaciones:
%\begin{align*}
%  \frac{\mathrm{}}{}
   %  \end{align*}
\newpage

\section{Definiciones}
\label{sec:defi}


\subsection{Distribución uniforme}

\begin{definition}
  Una variable aleatoria continua tiene \textit{distribución uniforme} en el intervalo $\left[a,b\right]$ si su función de densidad de probabilidad $f$ es dado por $f(x)=0$ si $x$ no está en $\left[a,b\right]$ y
  \begin{equation*}
    f(x)=\frac{1}{\beta-\alpha}\quad\text{para }\alpha \le x\le \beta.
  \end{equation*}
 Denotamos esta distribución por $U\left(\alpha,\beta\right)$.
\end{definition}

\subsection{La fórmula del cambio de variable}
\label{sec:change}

Con frecuencia no queremos calcular el valor esperado de una variable aleatoria $X$, pero %TODO
de una función $X$, como, por ejemplo, $X^2$. Entonces necesitamos determinar la distribución $Y=X^2$, por ejemplo para calcular la función de distribución $F_Y$ de $Y$ (este es un ejemplo del problema general de cómo las distribuciones bajo las transformaciones -- este tópico es es tema del capitulo 8). Para un ejemplo concreto, suponga un arquitecto quiere maximizar %TODO  Page 94
en el tamaño de los edificios: estos deben ser el mismo ancho y profunidad $X$, pero $X$ es uniformemente distribuido entre $0$ y $10$ metros. ¿Cuál es la distribución del área $X^2$ de un edificio? En particular, ¿será esta distribución (cualquier próxima a la) uniforme? Calculemos $F_Y$, para $0\le a\le 100$:
\begin{equation*}
  F_Y(a)=\mathds{P}\left(X^2\le a\right)=\mathds{P}\left(X\le\sqrt{a}\right)=\frac{\sqrt{a}}{10}.
\end{equation*}
Así, la función de densidad de probabilidad $f_Y$ para el área es, para $0<y<199$ metros cuadrados %TODO usar el paquete siunitx
, dado por
\begin{equation*}
  f_Y(y)=\diff{F_Y(y)}{y}=\diff{\sqrt{y}}{y}=\frac{1}{20\sqrt{y}}.
\end{equation*}
Esto significa que los edificios con áreas pequeñas son %TODO
, porque $f_Y$ explota cerca de 0--vea también la %TODO singularidad
, en el cual %TODO
$f_Y$.

Sorpresivamente, esto no es muy visible en %TODO
, un ejemplo donde debemos crear en nuestros cálculos más que en nuestros ojos. En la figura las ubicaciones de los edificios son generados por el proceso de Poisson, el tema del capítulo 12.

Suponga que un %TODO
tiene que hacer una ofera en el precio de las %TODO
de los edificios. El monto concreto que neceistará será proporcional al área $X^2$ de un edificio. Asi, su problema es: ¿cuál es el área esperada de un edificio? Con $f_Y$ de %TODO \eqref{7.1}
él encuentra
\begin{equation*}
  \mathds{E}\left[X^2\right]=\mathds{E}\left[Y\right]=\int_{0}^{199}\! y\cdot\frac{1}{20\sqrt{y}}\dl y=\int_{0}^{100}\frac{\sqrt{y}}{20}\dl y={\left[\frac{1}{20}\frac{2}{3}y^{\tfrac{3}{2}}\right]}_{0}^{100}=33\tfrac{1}{3}\mathrm{m}^2.
\end{equation*}
   %    ODO GRaficar con R la función dada.
La densidad de probabilidad del cuadrado de una variable aleatoria $U\left(0,10\right)$.

Es interesante notar que \emph{realmente} necesitamos hacer este cálculo, porque el valor esperado \emph{no} es simplemente el producto del ancho esperado y la profundidad esperada, que es % \SI{25}{\metre\per\square}

Sin embargo, existe una manera mucho más fácil en el cual el %TODO constructor
puede obtener este resultado. El podría tener el argumento que el valor del \emph{área} es $x^2$ cuando $x$ es el ancho, y que él debería tomar el peso  promedio de \emph{esos} valores, donde el peso de cada ancho $x$ es dado por el valor $f_X(x)$ de la densidad de probabilidad de $X$. Entonces él podría tener calculado
\begin{equation*}
  \mathds{E}\left[X^2\right]=\int_{-\infty}^{\infty}x^2f_X(x)\dl x=\int_{0}^{10}x^2\cdot\frac{1}{10}\dl x={\left[\frac{1}{30}x^3\right]}_{0}^{10}=33\tfrac{1}{3}%TODO.
\end{equation*}
Esto de hecho es un teorema matemático que esto es \emph{siempre} una manera correcta de calcular los valores esperados de funciones de variables aleatorias.

\begin{theorem}[La fórmula del cambio de variable]
  Sea $X$ una variable aleatoria, y sea $g\colon\mathds{R}\rightarrow\mathds{R}$ una función.

  Si $X$ es discreta, tomando los valores $a_1, a_2, \ldots$, entonces
  \begin{equation*}
    \mathds{E}\left[g\left(\\right]
  \end{equation*}
\end{theorem}

   %    TODO agregar defincion de dsitriucion  gamma

% TODO Justificar que la fincion exponencial se puede expresar como -1-lambda que tiene distribucion expoencial.

\end{document}
https://stackoverflow.com/questions/2199463/emacs-auctex-run-command-on-file-that-is-not-currently-open

\begin{lem}[Alinhac-Lerner {\cite[p.~37]{a-l}}]
ith the output

\begin{titlepage}
	\extratitle{
	\textbf{\LARGE Traducción del libro\\[\baselineskip]
	\textit{Una moderna introducción a la\linebreak Estadística y la Probabilidad}\\[\baselineskip]
	Entendiendo por qué y cómo}}

\title{El proceso de Poisson}
\author{F.M. Deeking}
\begin{spacing}{1}
	\maketitle
\end{spacing}
\thispagestyle{empty}
\end{titlepage}


\section{Definiciones}
\label{sec:defi}

\subsection{Distribución uniforme}

\begin{definition}
  Una variable aleatoria continua tiene \textit{distribución uniforme} en el intervalo $\left[a,b\right]$ si su función de densidad de probabilidad $f$ es dado por $f(x)=0$ si $x$ no está en $\left[a,b\right]$ y
  \begin{equation*}
    f(x)=\frac{1}{\beta-\alpha}\quad\text{para }\alpha \le x\le \beta.
  \end{equation*}
 Denotamos esta distribución por $U\left(\alpha,\beta\right)$.
\end{definition}

\subsection{La fórmula del cambio de variable}
\label{sec:change}

Con frecuencia no queremos calcular el valor esperado de una variable aleatoria $X$, pero %TODO
de una función $X$, como, por ejemplo, $X^2$. Entonces necesitamos determinar la distribución $Y=X^2$, por ejemplo para calcular la función de distribución $F_Y$ de $Y$ (este es un ejemplo del problema general de cómo las distribuciones bajo las transformaciones -- este tópico es es tema del capitulo 8). Para un ejemplo concreto, suponga un arquitecto quiere maximizar %TODO  Page 94
en el tamaño de los edificios: estos deben ser el mismo ancho y profunidad $X$, pero $X$ es uniformemente distribuido entre $0$ y $10$ metros. ¿Cuál es la distribución del área $X^2$ de un edificio? En particular, ¿será esta distribución (cualquier próxima a la) uniforme? Calculemos $F_Y$, para $0\le a\le 100$:
\begin{equation*}
  F_Y(a)=\mathds{P}\left(X^2\le a\right)=\mathds{P}\left(X\le\sqrt{a}\right)=\frac{\sqrt{a}}{10}.
\end{equation*}
Así, la función de densidad de probabilidad $f_Y$ para el área es, para $0<y<199$ metros cuadrados %TODO usar el paquete siunitx
, dado por
\begin{equation*}
  f_Y(y)=\diff{F_Y(y)}{y}=\diff{\sqrt{y}}{y}=\frac{1}{20\sqrt{y}}.
\end{equation*}
Esto significa que los edificios con áreas pequeñas son %TODO
, porque $f_Y$ explota cerca de 0--vea también la %TODO singularidad
, en el cual %TODO
$f_Y$.

Sorpresivamente, esto no es muy visible en %TODO
, un ejemplo donde debemos crear en nuestros cálculos más que en nuestros ojos. En la figura las ubicaciones de los edificios son generados por el proceso de Poisson, el tema del capítulo 12.

Suponga que un %TODO
tiene que hacer una oferta en el precio de las %TODO
de los edificios. El monto concreto que necesitará será proporcional al área $X^2$ de un edificio. Así, su problema es: ¿cuál es el área esperada de un edificio? Con $f_Y$ de %TODO \eqref{7.1}
él encuentra
\begin{equation*}
  \mathds{E}\left[X^2\right]=\mathds{E}\left[Y\right]=\int_{0}^{199}\! y\cdot\frac{1}{20\sqrt{y}}\dl y=\int_{0}^{100}\frac{\sqrt{y}}{20}\dl y={\left[\frac{1}{20}\frac{2}{3}y^{\tfrac{3}{2}}\right]}_{0}^{100}=33\tfrac{1}{3}\mathrm{m}^2.
\end{equation*}
   %    ODO GRaficar con R la función dada.
La densidad de probabilidad del cuadrado de una variable aleatoria $U\left(0,10\right)$.

Es interesante notar que \emph{realmente} necesitamos hacer este cálculo, porque el valor esperado \emph{no} es simplemente el producto del ancho esperado y la profundidad esperada, que es % \SI{25}{\metre\per\square}

Sin embargo, existe una manera mucho más fácil en el cual el %TODO constructor
puede obtener este resultado. El podría tener el argumento que el valor del \emph{área} es $x^2$ cuando $x$ es el ancho, y que él debería tomar el peso  promedio de \emph{esos} valores, donde el peso de cada ancho $x$ es dado por el valor $f_X(x)$ de la densidad de probabilidad de $X$. Entonces él podría tener calculado
\begin{equation*}
  \mathds{E}\left[X^2\right]=\int_{-\infty}^{\infty}x^2f_X(x)\dl x=\int_{0}^{10}x^2\cdot\frac{1}{10}\dl x={\left[\frac{1}{30}x^3\right]}_{0}^{10}=33\tfrac{1}{3}%TODO.
\end{equation*}
Esto de hecho es un teorema matemático que esto es \emph{siempre} una manera correcta de calcular los valores esperados de funciones de variables aleatorias.

\begin{theorem}[La fórmula del cambio de variable]
  Sea $X$ una variable aleatoria, y sea $g\colon\mathds{R}\rightarrow\mathds{R}$ una función.

  Si $X$ es discreta, tomando los valores $a_1, a_2, \ldots$, entonces
  \begin{equation*}
    \mathds{E}%\left[g\left(\right]
  \end{equation*}
\end{theorem}

% TODO Agregar la definición de la distribución Gamma.

% TODO Justificar el hecho que la función exponencial se puede expresar como -1\lambda que tiene la distribución exponencial.

\subsection{Ejercicios}

\begin{enumerate}
	\item En cada uno de los siguientes ejemplos, intente indicar cuándo el proceso de Poisson sería un buen modelo.
	\begin{enumerate}
		\item La cantidad de bancarrotas de las empresas en los Estados Unidos.
		\item Las  veces que un pollo pone huevos.
		\item Las veces que un avión se estrella en un registro mundial.
		\item Las ubicaciones de las palabras deletreadas en un libro.
		\item Las cantidad de los accidentes de tránsito en una intersección.
	\end{enumerate}
	\item La cantidad de clientes que visitan un banco en un día es modelado por la distribución de Poisson. Se sabe que la probabilidad de que no haya clientes es de $0.00001$. ¿Cuál es el número esperado de clientes?
	\item Sea $N$ una variable aleatoria con distribución $\mathrm{Poiss}(4)$. ¿Cuánto es $\mathds{P}\left(N=4\right)$?
	\item Sea $X$ una variable aleatoria con distribución $\mathrm{Poiss}(2)$. ¿Cuánto es $\mathds{P}\left(X\le1\right)$?
	\item El número de errores de un disco duro es modelado como una variable aleatoria de Poisson con la esperanza de un error en cada Mb, esto es \SI{100}{\mega\byte\per\second}, esto es, en cada $2^{20}$ bytes.
	\begin{enumerate}
		\item ¿Cuál es la probabilidad de que haya al menos un error en un sector de $512$ bytes?
		\item El disco duro es una unidad de disco de \SI{18.62}{\giga\byte} con \num{39054015} sectores. ¿Cuál es la probabilidad de que haya al menos un error en el disco duro?
	\end{enumerate}
	\item Una cierta marca de alambre de cobre tiene fallas en cada \SI{40}{\centi\metre}. Modele las  ubicaciones de las fallas como un proceso de Poisson. En un intervalo de tiempo de \SI{20}{\minute} es dividido en ranuras de tiempo de \SI{10}{\second}. En un cierto punto a lo largo de la autopista, el número de autos que pasan es registrado para cada ranura de \SI{19}{\second}. Sea $n_j$ el número de ranuras en las que $j$ autos han pasado para $j=0,\ldots,9$. Suponga que uno encuentra
	\centering
	\begin{tabular}{c|cccccccccc}
		$j$ 	& $0$ & $1$ & $2$ & $3$ & $4$ & $5$ & $6$ & $7$ & $8$ & $9$ \\
		\hline
		$f(x)$& $19$ & $38$ & $28$ & $20$ & $7$ & $3$ & $4$ & $0$ & $0$ & $1$
	\end{tabular}\quad.

	Note que el número total de autos que pasan en esos $20$ minutos es $230$.
	\begin{enumerate}
		\item ¿Qué elegirías para el parámetro de intensidad $\lambda$?\label{12.7a}
		\item Suponga que estima la probabilidad de que cero autos pasando en una ranura de tiempo de \SI{10}{\second} por $n_0$ dividido por el número total de ranuras de tiempo. ¿Eso está de acuerdo (razonablemente) con el valor que sigue desde su respuesta en~\ref{12.7a}? %TODO dequeismo
		\item  ¿Qué tomarías para la probabilidad que $10$ autos pasan en una ranura de tiempo de \SI{10}{\second}?
	\end{enumerate}
	\item Sea $X$ una variable aleatoria de Poisson con parámetro $\mu$.
	\begin{enumerate}
		\item Calcule $\mathds{E}\left[X\left(X-1\right)\right]$.
		\item Calcule $\mathrm{Var}\left(X\right)$, usando el siguiente hecho:
		\begin{equation*}
		\mathrm{Var}\left(X\right)=\mathds{E}\left[X\left(X-1\right)\right]+\mathds{E}\left[X\right]-{\left(\mathds{E}\left[X\right]\right)}^2.
		\end{equation*}
	\end{enumerate}
	\item Sea $Y_1$ y $Y_2$ dos variables aleatorias de Poisson independientes con parámetros $\mu_1$ y $\mu_2$, respectivamente. Muestre que $Y=Y_1+Y_2$ también es una distribución de Poisson. En vez de usar la regla de adicción en la Sección 11.1 como en el ejercicio 11.2, puede probar esto sin ningún cálculo considerando el número de puntos de un proceso de Poisson (con intensidad $1$) en dos intervalos disjuntos de longitud $\mu_1$ y $\mu_2$.
	\item Sea $X$ una variable aleatorias con una distribución $\mathrm{Pois}\left(\mu\right)$. Muestre lo siguiente. Si $\mu<1$, entonces las probabilidades $\mathds{P}\left(X=k\right)$ son primero crecientes, luego decreciente (confer ). ¿Qué ocurre si $\mu=1$? %TODO Figura
	\item Considere el proceso de Poisson unidimensional con intensidad $\lambda$. Muestre que el número de puntos en $\left[0,t\right]$, \emph{dado} que el número de puntos en $\left[0,2t\right]$ es igual a $n$, tiene una distribución $\mathrm{Bin}\left(n,\tfrac{1}{2}\right)$.

	\emph{Ayuda:} escriba el evento $\left\{N\left(\left[0,s\right]\right)=k,N\left(\left[0,2s\right]\right)=n\right\}$ como la intersección de eventos (¡independientes!) $\left\{N\left(\left[0,s\right]\right)=k\right\}$ y $\left\{N\left(\left[0,2s\right]\right)=n-k\right\}$.
	\item Consideremos el proceso de Poisson unidimensional. Suponga que para algunos $\alpha>0$ se da que hay exactamente dos puntos $\left[0,a\right]$, o en otras palabras: $N_\alpha=2$. El objetivo de este ejercicio es determinar la distribución conjunta de $X_1$ y $X_2$, las ubicaciones de los dos puntos, sujeto a $N_\alpha=2$.
	\begin{enumerate}
		\item Pruebe que para $0<s<t<\alpha$\label{12.12a}
		\begin{multline*}
			\mathds{P}\left(X_1\le X_2\le t, N_\alpha=2\right) \\
			=\mathds{P}\left(X_2\le t, N_\alpha=2\right)-\mathds{P}\left(X_1>s,X_2\le t, N_\alpha=2\right).
		\end{multline*}
		\item Deduzca de~\ref{12.12a} que\label{12.12b}
		\begin{equation*}
			\mathds{P}\left(X_1\le s,X_2\le t,N_\alpha=2\right)=e^{-\lambda\alpha}\left(\frac{\lambda^2t^2}{2!}-\frac{\lambda^2{\left(t-s\right)}^2}{2!}\right).
		\end{equation*}
		\item Deduzca de~\ref{12.12b} que $0<s<t<\alpha$
		\begin{equation*}
			\mathds{P}\left(X_1\le s, X_2\le t\mid N_\alpha=2\right)=\frac{t^2-{\left(t-s\right)}^2}{\alpha^2}.
		\end{equation*}
	\end{enumerate}
	\item Al caminar por un prado encontramos dos tipos de flores, margaritas y dientes de león. Mientras caminamos en línea recta, modelamos las posiciones de las flores que encontramos con un proceso de Poisson con intensidad $\lambda$. Parece que aproximadamente una de cada cuatro flores es margarita. Olvidando los dientes de león, ¿cómo se ve el proceso de las \emph{margaritas}? Esta pregunta será respondida con los siguientes pasos:
	\begin{enumerate}
		\item Sea $N_t$ el número total de flores, $X_t$ el número de margaritas, e $Y_t$ el número de dientes de león que encontramos durante los primeros $t$ minutos de nuestro paseo. Note que $X_t+Y_t=N_t$. Suponga que cada flor es un margarita con probabilidad $1/4$, independiente de las otras flores. Argumenta que\label{12.13a}
		\begin{equation*}
			\mathds{P}\left(X_t=n,Y_t=m\mid N_t=n+m\right)=\binom{n+m}{n}{\left(\frac{1}{4}\right)}^n{\left(\frac{3}{4}\right)}^m.
		\end{equation*}
		\item Muestre que
		\begin{equation*}
			\mathds{P}\left(X_t=n,Y_t=m\right)=\frac{1}{n!}\frac{1}{m!}{\left(\frac{1}{4}\right)}^n{\left(\frac{3}{4}\right)}^me^{-\lambda t}{\lambda t}^{n+m},
		\end{equation*}
		condicionando sobre $N_t$ y usando~\ref{12.13a}.
		\item Escribiendo $e^{-\lambda t}=e^{-\left(\lambda/4\right)t}e^{-\left(3\lambda/4\right)t}$ y sumando sobre $m$, muestre que\label{12.13c}
		\begin{equation*}
			\mathds{P}\left(X_t=n\right)=\frac{1}{n!}e^{-\left(\lambda/4\right)t}{\left(\frac{\lambda t}{4}\right)}^n.
		\end{equation*}
		Ya que está claro que los números de margaritas que encontramos en intervalos de tiempo disjuntos son independientes, podríamos concluir de~\ref{12.13c} que el proceso $\left(X_t\right)$ es de nuevo un proceso de Poisson, con intensidad $\lambda/4$. Uno dice a menudo que el proceso $\left(X_t\right)$ es obtenido por \emph{adelgazamiento} del proceso $\left(N_t\right)$. En nuestro ejemplo este corresponde a recoger todos los dientes de león.
	\end{enumerate}
	\item En este ejercicio nos fijamos en un ejemplo simple de variables aleatorias $X_n$ que tiene la propiedad que sus distribuciones convergen a las distribuciones de una variable aleatorias $X$ cuando $n\to\infty$. Sea para $n=1,2,\ldots$ las variables aleatorias definidas por
	\begin{equation*}
	P\left(X_n=0\right)=1-\frac{1}{n}\quad\text{y}\quad\mathds{P}\left(X_n=7n\right)=\frac{1}{n}.
	\end{equation*}
	\begin{enumerate}
		\item Sea $X$ una variable aleatorias que es igual a $0$ con probabilidad $1$. Muestre que para todo $a$, las funciones de masa de probabilidad $p_{X_m}(a)$ de $X_n$ converge a la función de probabilidad de masa $p_X(a)$ de $X$ cuando $n\to\infty$. Note que $\mathds{E}\left[X\right]=0$.
		\item Muestre que sin embargo $\mathds{E}\left[X_n\right]=7$ para todo $n$.
	\end{enumerate}
\end{enumerate}
%%% Local Variables:
%%% mode: latex
%%% TeX-master: "../main"
%%% End:

%% --------------------
%% |   Bibliography   |
%% --------------------
%% Add entry to the table of contents for the bibliography

\vfill                                                                           
\nocite{*}                                                                       
\printbibliography[title={Referencias bibliográficas},heading=bibintoc]

\end{document}
https://stackoverflow.com/questions/2199463/emacs-auctex-run-command-on-file-that-is-not-currently-open

\begin{lem}[Alinhac-Lerner {\cite[p.~37]{a-l}}]
ith the output