% arara: clean: {extensions: ['log','aux','bbl','bcf','blg','run.xml','fls','fdb_latexmk','synctex.gz','toc']}
% arara: xelatex: {shell: yes, synctex: yes}
% arara: xelatex: {shell: yes, synctex: yes}
% arara: clean: {extensions: ['log','aux','bbl','bcf','blg','run.xml','fls','fdb_latexmk','synctex.gz','toc']}
\documentclass[
	10pt,
	twoside=false,
	twocolumn=false,
	BCOR=12mm,
	DIV=calc
]{scrartcl} %,headlines=2.1,footlines=2.1
%\usepackage[paper=a4,pagesize]{typearea} %headinclude=false,footinclude=false
\usepackage[onehalfspacing]{setspace}
\AfterTOCHead{\singlespacing}
\KOMAoptions{DIV=last}

\usepackage[T1]{fontenc}
\usepackage[spanish,es-sloppy]{babel}
	\spanishdatedel
\usepackage{libertine}
\usepackage[lite]{mtpro2}
\usepackage{amsthm}
\usepackage[libertine,scaled=1.1]{newtxmath}
\usepackage[scaled=.92]{sourcesanspro}
\usepackage[scaled=.78]{beramono}
\usepackage[ISO]{diffcoeff}
\usepackage{dsfont}
\usepackage{siunitx}
	\sisetup{binary-units=true,per-mode=symbol}
\usepackage[citestyle=numeric,style=numeric,backend=biber]{biblatex}
\addbibresource{./bibliography/bib.bib}
\addbibresource{bib.bib}
\theoremstyle{definition}
\newtheorem{definition}[section]{Definición}
\newtheorem{theorem}[section]{Teorema}

\setlength{\marginparwidth}{1.5\marginparwidth}

\begin{document}

\tableofcontents

\begin{titlepage}
	\extratitle{\textbf{\LARGE Traducción del artículo\\[\baselineskip] \textit{Las variaciones de la probabilidad en la\linebreak distribución de partículas $\alpha$}}}
	\title{Sobre la distribución de probabilidad de las partículas $\alpha$}
	\author{Harry Bateman}
	\begin{spacing}{1}
		\maketitle
	\end{spacing}
	\thispagestyle{empty}
\end{titlepage}

\documentclass[10pt,twoside=false,twocolumn=false,BCOR=12mm,DIV=calc]{scrartcl} %,headlines=2.1,footlines=2.1
%\usepackage[paper=a4,pagesize]{typearea} %headinclude=false,footinclude=false
\usepackage[onehalfspacing]{setspace}
\AfterTOCHead{\singlespacing}
\KOMAoptions{DIV=last}
\usepackage[T1]{fontenc}
\usepackage[spanish,es-sloppy]{babel}
\spanishdatedel
\usepackage{libertine}
\usepackage[lite]{mtpro2}
\usepackage{amsthm}
\usepackage[libertine,scaled=1.1]{newtxmath}
\usepackage[ISO]{diffcoeff}
% \providecommand{\dt}{\mathrm{d}t}
\usepackage{dsfont}
\usepackage{siunitx}

\theoremstyle{definition}
\newtheorem{definition}[section]{Definición}
\newtheorem{theorem}[section]{Teorema}

\setlength{\marginparwidth}{1.5\marginparwidth}

\begin{document}

\begin{titlepage}
  \extratitle{\textbf{\LARGE Traducción del artículo\\[\baselineskip] \textit{Las variaciones de la probabilidad en la distribución de partículas $\alpha$}}}
  \title{Sobre la distribución de probabilidad de las partículas $\alpha$}
  \author{H. Bateman}
  \begin{spacing}{1}
    \maketitle
  \end{spacing}
  \thispagestyle{empty}
\end{titlepage}

\tableofcontents
\newpage
\section{Una nota}
\label{sec:note}


Sea $\lambda t$ la probabilidad que una partícula $\alpha$ golpee la pantalla en un intervalo de tiempo $\dl t$. Si los intervalos de tiempo bajo la consideración son pequeños comparados con el período de tiempo de la sustancia radiactiva, podemos asumir que $\lambda$ de $t$. Ahora, sea $W_n(t)$ la probabilidad que $n$ partículas $\alpha$ golpeen la pantalla en un intervalo de tiempo $t$, entonces la probabilidad que $(n+1)$ partículas golpeen la pantalla en un intervalo $t+\mathrm{d}t$ es la suma de las dos probabilidades. En primer lugar, las $n+1$ partículas $\alpha$ pueden golpear la pantalla en el intervalo $t$ y ninguno en el intervalo $\mathrm{d}t$. La probabilidad de que esto pueda ocurrir es $(1-\lambda t\mathrm{d}t)W_{n+1}(t)$. En segundo lugar, $n$ partículas $\alpha$ podrían golpear la pantalla en el intervalo $t$ y una partícula en el intervalo $\mathrm{d}t$; la probabilidad de que esto ocurra es $\lambda\mathrm{d}t W_n(t)$. Por lo tanto
\begin{equation*}
  W_{n+1}\left(t+\mathrm{d}t\right)=\left(1-\lambda\mathrm{d}t\right)W_{n+1}(t)+\lambda\mathrm{d}t W_n\left(t\right).
\end{equation*}
Procediendo al límite, tenemos
\begin{equation*}
  \frac{\mathrm{d}W_{n+1}}{\mathrm{d}t}=\lambda\left(W_n-W_{n+1}\right).
\end{equation*}
Poniendo $n=0,1,2,\ldots$ en una sucecsión tenemos el sistema de ecuaciones:
%\begin{align*}
%  \frac{\mathrm{}}{}
   %  \end{align*}
\newpage

\section{Definiciones}
\label{sec:defi}


\subsection{Distribución uniforme}

\begin{definition}
  Una variable aleatoria continua tiene \textit{distribución uniforme} en el intervalo $\left[a,b\right]$ si su función de densidad de probabilidad $f$ es dado por $f(x)=0$ si $x$ no está en $\left[a,b\right]$ y
  \begin{equation*}
    f(x)=\frac{1}{\beta-\alpha}\quad\text{para }\alpha \le x\le \beta.
  \end{equation*}
 Denotamos esta distribución por $U\left(\alpha,\beta\right)$.
\end{definition}

\subsection{La fórmula del cambio de variable}
\label{sec:change}

Con frecuencia no queremos calcular el valor esperado de una variable aleatoria $X$, pero %TODO
de una función $X$, como, por ejemplo, $X^2$. Entonces necesitamos determinar la distribución $Y=X^2$, por ejemplo para calcular la función de distribución $F_Y$ de $Y$ (este es un ejemplo del problema general de cómo las distribuciones bajo las transformaciones -- este tópico es es tema del capitulo 8). Para un ejemplo concreto, suponga un arquitecto quiere maximizar %TODO  Page 94
en el tamaño de los edificios: estos deben ser el mismo ancho y profunidad $X$, pero $X$ es uniformemente distribuido entre $0$ y $10$ metros. ¿Cuál es la distribución del área $X^2$ de un edificio? En particular, ¿será esta distribución (cualquier próxima a la) uniforme? Calculemos $F_Y$, para $0\le a\le 100$:
\begin{equation*}
  F_Y(a)=\mathds{P}\left(X^2\le a\right)=\mathds{P}\left(X\le\sqrt{a}\right)=\frac{\sqrt{a}}{10}.
\end{equation*}
Así, la función de densidad de probabilidad $f_Y$ para el área es, para $0<y<199$ metros cuadrados %TODO usar el paquete siunitx
, dado por
\begin{equation*}
  f_Y(y)=\diff{F_Y(y)}{y}=\diff{\sqrt{y}}{y}=\frac{1}{20\sqrt{y}}.
\end{equation*}
Esto significa que los edificios con áreas pequeñas son %TODO
, porque $f_Y$ explota cerca de 0--vea también la %TODO singularidad
, en el cual %TODO
$f_Y$.

Sorpresivamente, esto no es muy visible en %TODO
, un ejemplo donde debemos crear en nuestros cálculos más que en nuestros ojos. En la figura las ubicaciones de los edificios son generados por el proceso de Poisson, el tema del capítulo 12.

Suponga que un %TODO
tiene que hacer una ofera en el precio de las %TODO
de los edificios. El monto concreto que neceistará será proporcional al área $X^2$ de un edificio. Asi, su problema es: ¿cuál es el área esperada de un edificio? Con $f_Y$ de %TODO \eqref{7.1}
él encuentra
\begin{equation*}
  \mathds{E}\left[X^2\right]=\mathds{E}\left[Y\right]=\int_{0}^{199}\! y\cdot\frac{1}{20\sqrt{y}}\dl y=\int_{0}^{100}\frac{\sqrt{y}}{20}\dl y={\left[\frac{1}{20}\frac{2}{3}y^{\tfrac{3}{2}}\right]}_{0}^{100}=33\tfrac{1}{3}\mathrm{m}^2.
\end{equation*}
   %    ODO GRaficar con R la función dada.
La densidad de probabilidad del cuadrado de una variable aleatoria $U\left(0,10\right)$.

Es interesante notar que \emph{realmente} necesitamos hacer este cálculo, porque el valor esperado \emph{no} es simplemente el producto del ancho esperado y la profundidad esperada, que es % \SI{25}{\metre\per\square}

Sin embargo, existe una manera mucho más fácil en el cual el %TODO constructor
puede obtener este resultado. El podría tener el argumento que el valor del \emph{área} es $x^2$ cuando $x$ es el ancho, y que él debería tomar el peso  promedio de \emph{esos} valores, donde el peso de cada ancho $x$ es dado por el valor $f_X(x)$ de la densidad de probabilidad de $X$. Entonces él podría tener calculado
\begin{equation*}
  \mathds{E}\left[X^2\right]=\int_{-\infty}^{\infty}x^2f_X(x)\dl x=\int_{0}^{10}x^2\cdot\frac{1}{10}\dl x={\left[\frac{1}{30}x^3\right]}_{0}^{10}=33\tfrac{1}{3}%TODO.
\end{equation*}
Esto de hecho es un teorema matemático que esto es \emph{siempre} una manera correcta de calcular los valores esperados de funciones de variables aleatorias.

\begin{theorem}[La fórmula del cambio de variable]
  Sea $X$ una variable aleatoria, y sea $g\colon\mathds{R}\rightarrow\mathds{R}$ una función.

  Si $X$ es discreta, tomando los valores $a_1, a_2, \ldots$, entonces
  \begin{equation*}
    \mathds{E}\left[g\left(\\right]
  \end{equation*}
\end{theorem}

   %    TODO agregar defincion de dsitriucion  gamma

% TODO Justificar que la fincion exponencial se puede expresar como -1-lambda que tiene distribucion expoencial.

\end{document}
https://stackoverflow.com/questions/2199463/emacs-auctex-run-command-on-file-that-is-not-currently-open

\begin{lem}[Alinhac-Lerner {\cite[p.~37]{a-l}}]
ith the output

\begin{titlepage}
	\extratitle{
	\textbf{\LARGE Traducción del libro\\[\baselineskip]
	\textit{Una moderna introducción a la\linebreak Estadística y la Probabilidad}\\[\baselineskip]
	Entendiendo por qué y cómo}}

\title{El proceso de Poisson}
\author{F.M. Deeking}
\begin{spacing}{1}
	\maketitle
\end{spacing}
\thispagestyle{empty}
\end{titlepage}

\section{El proceso de Poisson}\label{sec:12}

	En varios fenómenos aleatorios encontramos, no solo uno o dos variables aleatorias que juegan un rol, sino una colección completa. En este caso frecuentemente habla de un \textit{proceso} aleatorio. El proceso de Poisson es un tipo simple de proceso aleatorio, que modela la ocurrencia de puntos aleatorios en el espacio o tiempo. Existen números formas en el que proceso de puntos aleatorios aumenta: algunos ejemplos son presentados en la primera sección. El proceso de Poisson describe en cierto sentido la \textit{forma más aleatoria} para distribuir puntos en el espacio o tiempo. Este es hecho más preciso con las nociones de homogeneidad e independencia.

\subsection{Los puntos aleatorios}

	Ejemplos típicos de la ocurrencia de puntos aleatorios son: tiempos de llegadas de mensajes de correo electrónico en un servidor, las veces en que los asteroides golpean la tierra, tiempos de llegada de materiales radiactivos en un contador Geiger\footnote{Un contador Geiger es un instrumento que permite medir la radiactividad de un objeto o lugar.}, veces en el que su computadora falla y los tiempos en que fallan los componentes electrónicos y los tiempos de llegada de personas a una bomba de agua en un oasis.

	Ejemplos de la ocurrencia de los puntos aleatorios son: las ubicaciones de los impactos de asteroides con la tierra (dimensión $2$), las ubicaciones de las imperfecciones un material (dimensión $3$), y las ubicaciones de los árboles en un bosque. (dimensión $2$).

	Algunos de estos fenómenos son mejor modelados por el proceso de Poisson que por otros. De forma aproximada, uno podría decir que el proceso de Poisson se aplica con frecuentemente a situaciones donde existe una población muy grande, y cada miembro de este tiene una pequeña probabilidad de producir un punto del proceso. Esto es, por ejemplo, bien satisfecho en el ejemplo del contador Geiger donde, en una descomunal colección de átomos, solo unos pocos emitirán una partícula radiactiva. Una propiedad del proceso de Poisson--como veremos en breve--es que los puntos pueden estar arbitrariamente juntos. Por lo tanto, las ubicaciones de los árboles no están tan bien modeladas por el proceso de Poisson.

\subsubsection{Echando un vistazo más de cerca las llegadas aleatorias}

	Un ejemplo bien conocido que es usualmente modelado por el proceso de Poisson es que las llamadas que llegan a una central telefónica--la central está conectada a un gran número de personas que hacen llamadas telefónicas de vez en cuando. Este será nuestro ejemplo principal en esta sección.

	\noindent Las llamadas telefónicas llegan en tiempos aleatorios $X_{1}, X_{2}, \ldots$ a la central telefónica durante el tiempo de tiempo $\left[0,t\right]$.

	\begin{figure}[!ht]
		\centering
		\includegraphics[width=0.7\paperwidth]{img12_1a}
		\caption*{}
	\end{figure}

	\noindent Los dos supuestos básicos que hacemos en estas llamadas aleatorias son
	\begin{description}
		\item[Homogeneidad] La tasa $\lambda$ a la que se producen las llamadas es constante a través del tiempo: en un subintervalo de longitud $u$ la esperanza del número de llamadas telefónicas es $\lambda u$.
		\item[Independencia] El número de llegadas en intervalos de tiempo disjuntos son variables aleatorias independientes.
	\end{description}

	La homogeneidad es también llamado \textit{estacionario débil}. Denotamos el número total de llamadas en un intervalo $I$ por $N\left(I\right)$, abreviando $N\left(\left[0,t\right]\right)$ por $N_{t}$. Entonces, la homogeneidad implica que requerimos
	\[
		\mathds{E}\left[N_{t}\right]=\lambda t.
	\]

	\begin{figure}[!ht]
		\centering
		\includegraphics[width=0.7\paperwidth]{img12_1b}
		\caption*{}
	\end{figure}

	\noindent Para obtener la distribución de $N_{t}$, dividimos el intervalo $\left[0,t\right]$ en $n$ intervalos de longitud $\tfrac{t}{n}$. Cuando $n$ es lo suficientemente grande, cada intervalo $I_{j,n}=\left(\tfrac{\left(j-1\right)t}{n},\tfrac{jt}{n}\right]$ contendrá $0$ o $1$ llegada: Para un $n$ tan grande (que también satisface $n>\lambda t$), sea $R_{j}$ el número de llegadas en el intervalo de tiempo $I_{j,n}$. Como $R_{j}$ es $0$0 o $1$, $R_{j}$ tiene una distribución $\mathrm{Ber}\left(p_{j}\right)$ para algunos $p_{j}$. Recuerde que para una variable aleatoria de Bernoulli, $\mathds{E}\left[R_{j}\right]=0\cdot\left(1-p_{j}\right)+1\cdot p_{j}=p_{j}$. Por el supuesto de homogeneidad, para cada $j$
	\[
		p_{j}=\lambda\cdot\text{longitud de }I_{j,n}=\frac{\lambda t}{n}.
	\]
	Sumando el número de llamadas en los intervalos nos da el número total de llamadas, así
	\[
		N_{t}=R_{1}+R_{2}+\cdots+R_{n}.
	\]
	Por el supuesto de independencia, los $R_{j}$ son variables aleatorias independientes, por lo tanto, $N_{t}$ tiene una distribución $\mathrm{Bin}\left(n,p\right)$ con $p=\tfrac{\lambda t}{n}$.

	\begin{definition}
		Una variable aleatoria discreta $X$ tiene una \textit{distribución de Poisson} con parámetro $\mu$, donde $\mu>0$ si su función de masa de probabilidad $p$ es dado por
	\[
		p\left(k\right)=\mathds{P}\left(X=k\right)=\frac{\mu^{k}e^{-\mu}}{k!}\quad\text{para }k=0,1,2,\ldots.
	\]
	Denotamos esta distribución por $\mathrm{Pois}\left(\mu\right)$.
	\end{definition}

	\begin{figure}[!ht]
		\centering
		\includegraphics[width=0.7\paperwidth]{img12_1}
		\caption*{Las funciones de masa de probabilidad de $\mathrm{Pois}\left(\num{0.9}\right)$ y las distribución $\mathrm{Pois}\left(\num{5}\right)$.}
	\end{figure}

	\begin{theorem}[\textsc{La esperanza y la varianza de una distribución de Poisson}]
		Sea $X$ una variable aleatoria con distribución de Poisson con parámetro $\mu$, entonces
		\[
			\mathds{E}\left[X\right]=\mu\quad\text{y}\quad\mathrm{Var}\left(X\right)=\mu.
		\]
	\end{theorem}

\subsection{El proceso de Poisson unidimensional}

	Derivaremos algunas propiedades de la sucesión de puntos aleatorios $X_{1},X_{2},\ldots$ que hemos considerado en la sección anterior. Lo que derivamos hasta ahora es que para cualquier intervalo $\left(s,s+t\right]$ el número $N\left(\left(s,s+t\right]\right)$ de puntos $X_{i}$ en ese intervalo es una variable aleatoria con una distribución $\mathrm{Pois}\left(\lambda t\right)$.

\subsubsection{Tiempos de llegada}

	Las diferencias
	\[
		T_{i}=X_{i}-X_{i-1}
	\]
	se llaman tiempos de llegada. Aquí definimos $T_{1}=X_{1}$, el tiempo de la primera llegada. Para determinar la distribución de probabilidad de $T_{1}$, observamos que el evento $\left\{T_{1}>t\right\}$ que la primera llamada de llegada después del tiempo $t$ es el mismo que el evento $\left\{N_{t}=0\right\}$ que no se ha realizado ninguna llamada en $\left[0,t\right]$. Pero esto implica que
	\[
		\mathds{P}\left(T_{1}\le t\right)=1-\mathds{P}\left(T_{1}>t\right)=1-\mathds{P}\left(N_{t}=0\right)=1-e^{-\lambda t}.
	\]
	Por lo tanto, $T_{1}$ tiene una distribución exponencial con parámetro $\lambda$.

	Para calcular la distribución conjunta de $T_{1}$ y $T_{2}$, consideramos la probabilidad condicional de que $T_{2}>t$, dado que $T_{1}=s$, y usamos la propiedad de que las llegadas en intervalos diferentes son independientes:
	\begin{align*}
		\mathds{P}\left(T_{2}>t\mid T_{1}=s\right)
		&=\mathds{P}\left(\text{ninguna llegada en }\left(s,s+t\right]\mid T_{1}=s\right)\\
		&=\mathds{P}\left(\text{ninguna llegada en }\left(s,s+t\right]\right)\\
		&=\mathds{P}\left(N\left(\left(s,s+t\right]\right)=0\right)=e^{-\lambda t}.
	\end{align*}
	Como esta respuesta no depende de $s$, concluimos que $T_{1}$ y $T_{2}$ son independientes, y
	\[
		\mathds{P}\left(T_{2}>t\right)=e^{-\lambda t},
	\]
	es decir, $T_{2}$ también es una distribución exponencial con parámetro $\lambda$. En realidad, aunque la conclusión es correcta, el método para derivarlo no lo es, porque condicionamos el evento $\left\{T_{1}=s\right\}$ que tiene probabilidad cero, Este problema puede evitarse condicionando el hecho de que el evento $T_{1}$ se encuentra en un pequeño intervalo, pero eso no se hará aquí. Análogamente, se puede mostrar que los $T_{i}$ son independientes y que tienen una distribución $\mathrm{Exp}\left(\lambda\right)$.. Esta bonita propiedad permite dar una definición simple del proceso de Poisson unidimensional.
	\begin{definition}
		El \textit{proceso de Poisson} unidimensional con intensidad $\lambda$ es una sucesión de variables aleatorias $X_{1},X_{2},X_{3},\ldots$ son variables aleatorias independientes, cada una con una distribución de $\mathrm{Exp}\left(\lambda\right)$.
	\end{definition}
	Tenga en cuenta que la conexión con $N_{t}$ es la siguiente: $N_{t}$ es igual al número de $X_{1}$ que son menores (o igual) que $t$.
	\begin{definition}[\textsc{Los puntos del proceso de Poisson}]
		Para $i=1,2,\ldots$ la variable aleatoria $X_{i}$ tiene una distribución $\mathrm{Gam}\left(i,\lambda\right)$.
	\end{definition}

\subsubsection{La distribución de los puntos}

	Otra interesante pregunta es: si sabemos que $n$ puntos se generan en un intervalo, ¿dónde se encuentran estos puntos? Dado que la distribución del número de puntos solo depende de la longitud del intervalo, y no de su ubicación, basta con determinar esto para un intervalo que comience en $0$. Sea este $\left[0,a\right]$ un intervalo. Comenzamos con el caso más simple, donde existe un solo punto en $\left[0,a\right]$: suponga que $N\left(\left[0,a\right]\right)=1$. Luego, para $0<s<a$:
	\begin{align*}
		\mathds{P}\left(X_{1}\le s\mid N\left(\left[0,a\right]\right)=1\right)
		&=\frac{\mathds{P}\left(X_{1}\le s,N\left(\left[0,a\right]\right)=1\right)}{\mathds{P}\left(N\left(\left[0,a\right]\right)=1\right)}\\
		&=\frac{\mathds{P}\left(N\left(\left[0,s\right]\right)=1,N\left(\left(s,a\right]\right)=0\right)}{\mathds{P}\left(N\left(\left[0,a\right]\right)=1\right)}\\
		&=\frac{\lambda se^{-\lambda s}e^{-\lambda\left(a-s\right)}}{\lambda ae^{-\lambda a}}\\
		&=\frac{s}{a}.
	\end{align*}
	Encontramos que condicionar el evento $\left\{N\left(\left[0,a\right]\right)=1\right\}$, la variable aleatoria $X_{1}$ está uniformemente distribuida sobre el intervalo $\left[0,a\right]$. Supongamos ahora que hay dos puntos en $\left[0,a\right]\colon N\left(\left[0,a\right]\right)=2$. De una manera similar a lo que hicimos para un punto, podemos mostrar que
	\[
		\mathds{P}\left(X_{1}\le s, X_{2}\le t\mid N\left(\left[0,a\right]=2\right)\right)=\frac{t^{2}-{\left(t-s\right)}^{2}}{a^{2}}.
	\]
	Ahora, recuerde el resultado del ejercicio%TODO 9.17
	si $U_{1}$ y $U_{2}$ son dos variables aleatorias independientes, ambas distribuidas uniformemente sobre $\left[0,a\right]$, entonces la función de distribución conjunta de $V=\min\left(U_{1},U_{2}\right)$ y $Z=\max\left(U_{1},U_{2}\right)$ está dado por
	\[
		\mathds{P}\left(V\le s, Z\le t\right)=\frac{t^{2}-{\left(t-s\right)}^{2}}{a^{2}}\quad\text{para }0\le s\le t\le a.
	\]
	Por lo tanto, hemos encontrado que, si olvidamos su orden, los dos puntos $\left[0,a\right]$ son independientes y se distribuyen uniformemente sobre $\left[0,a\right]$. Con algo más de trabajo, esto se generaliza a un número arbitrario de puntos, y llegamos a la siguiente propiedad:
	\begin{definition}[\textsc{Ubicación de los puntos, dado su número}]
		Dado que el proceso de Poisson tiene $n$ puntos en el intervalo $\left[a,b\right]$, las ubicaciones de estos puntos son independientemente distribuidas, cada una con una distribución uniforme en $\left[a,b\right]$.
	\end{definition}

\subsection{Procesos de Poisson multidimensionales}

	Nuestra definición del procesos de Poisson unidimensional, que comienza con los tiempos de llegadas, no se generaliza fácilmente, porque se basa en el ordenamiento de los números reales. Sin embargo, podemos ampliar fácilmente los supuestos de independencia, homogeneidad y la propiedad de la distribución de Poisson. Para hacer esto necesitamos una versión de mayor dimensión del concepto de longitud. Denotamos el volumen $k$--dimensional de un conjunto $A$ en un espacio $k$--dimensional por $\mathrm{m}\left(A\right)$ es el volumen de $A$.
	\begin{definition}
		El \textit{proceso de Poisson} $k$--dimensional con intensidad $\lambda$ es una colección de puntos aleatorios $X_{1}, X_{2}, X_{3},\ldots$ que tienen la propiedad que si $N\left(A\right)$ denota el número de puntos en el conjunto $A$, entonces
		\begin{description}
			\item[Homogeneidad] La variable aleatoria $N\left(A\right)$ tiene una distribución de Poisson con parámetro $\lambda\mathrm{m}\left(A\right)$.
			\item[Independencia] Para conjuntos disjuntos $A_{1}, A_{2},\ldots, A_{n}$ las variables aleatorias $N\left(A_{1},\right),N\left(A_{2},\right),\ldots,N\left(A_{n},\right)$ son independientes.
		\end{description}
	\end{definition}
	En la Figura~\ref{fig:7.4}, las ubicaciones de los edificios que el arquitecto quería distribuir en un terreno de \num{100}$\times$\SI{300}{\metre} se generaron mediante un proceso de Poisson bidimensional. Esto se ha hecho de la siguiente manera, Se puede volver a mostrar que dado el número total de puntos en un conjunto, estos puntos se distribuyen uniformemente sobre el conjunto $A$. Esto lleva al siguiente procedimiento: primero se genera un valor $n$ a partir de una distribución de Poisson con el parámetro apropiado $\left(\lambda\cdot\text{área}\right)$, entonces uno genera $n$ veces un punto distribuido uniformemente sobre el rectángulo de \num{100}$\times$\SI{300}{\metre}.

	En realidad, se puede generar un proceso de Poisson de mayor dimensión de una manera muy similar a la forma natural en que se puede hacer para el proceso de una dimensión. Directamente a partir de la definición del proceso unidimensional, vemos que puede obtenerse generando puntos consecutivamente con ``huecos'' distribuidos exponencialmente. Explicaremos un procedimiento similar para la dimensión dos. Para $s>0$, sea
	\[
		M_{s}=N\left(C_{s}\right),
	\]
	donde $C_{s}$ es la región circular de radio $s$, centrado en el origen. Como la área de $C_{s}$ es $\pi s^{2}$. Sea $R_{i}$ la distancia del $i$--ésimo punto más cerca al origen. Esto es ilustrado en la Figura~\ref{fig:12.2}


	Note que $R_{i}$ es el análogo al $i$--ésimo tiempo de llegada para el proceso de Poisson unidimensional: tenemos de hecho que
	\[
		R_{i}\le s\qquad\text{si y solo si}\qquad M_{s}\le i.
	\]
	En particular, con $i=1$ y $s=\sqrt{t}$,
	\[
		\mathds{P}\left(_{1}^{2}\le t\right)=\mathds{P}\left(R_{1\le\sqrt{t}}\right)=\mathds{P}\left(M_{\sqrt{t}}\le i\right).
	\]

	\begin{figure}[!ht]
		\centering
		\includegraphics[width=0.7\paperwidth]{img12_2}
		\caption{El proceso de Poisson en el plano con dos regiones circulares de los dos puntos más cercanos al origen.}
		\label{fig:12.2}
	\end{figure}

	Así,
	\[
		\mathds{P}\left(R_{i}^{2}\le t\right)=1-e^{-\lambda\pi t}\sum_{j=0}^{i-1}\frac{{\left(\lambda\pi t\right)}^{j}}{j!},
	\]
	lo que significa que $R_{i}^{2}$ tiene una distribución $\mathrm{Gam}\left(i,\lambda\pi\right)$--como se vio en la página 157 %TODO \pageref.
	Ya que las distribuciones gamma surgen como sumas de distribuciones exponenciales independientes, podemos también escribir
	\[
		R_{i}^{2}=R_{i-1}^{2}+T_{i},
	\]
	donde los $T_{i}$ son variables aleatorias independientes de $\mathrm{Exp}\left(\lambda\pi\right)$ (y donde $R_{0}=0$). Tenga en cuenta que esto es bastante similar al caso unidimensional. Para similar el proceso de Poisson bidimensional a partir de una sucesión de las variables aleatorias independientes $U_{1},U_{2},\ldots$ de $U\left(0,1\right)$, por lo tanto, se puede proceder de la siguiente manera


\section{Definiciones}
\label{sec:defi}

\subsection{Distribución uniforme}

\begin{definition}
  Una variable aleatoria continua tiene \textit{distribución uniforme} en el intervalo $\left[a,b\right]$ si su función de densidad de probabilidad $f$ es dado por $f(x)=0$ si $x$ no está en $\left[a,b\right]$ y
  \begin{equation*}
    f(x)=\frac{1}{\beta-\alpha}\quad\text{para }\alpha \le x\le \beta.
  \end{equation*}
 Denotamos esta distribución por $U\left(\alpha,\beta\right)$.
\end{definition}

\subsection{La fórmula del cambio de variable}
\label{sec:change}

Con frecuencia no queremos calcular el valor esperado de una variable aleatoria $X$, pero %TODO
de una función $X$, como, por ejemplo, $X^2$. Entonces necesitamos determinar la distribución $Y=X^2$, por ejemplo para calcular la función de distribución $F_Y$ de $Y$ (este es un ejemplo del problema general de cómo las distribuciones bajo las transformaciones -- este tópico es es tema del capitulo 8). Para un ejemplo concreto, suponga un arquitecto quiere maximizar %TODO  Page 94
en el tamaño de los edificios: estos deben ser el mismo ancho y profunidad $X$, pero $X$ es uniformemente distribuido entre $0$ y $10$ metros. ¿Cuál es la distribución del área $X^2$ de un edificio? En particular, ¿será esta distribución (cualquier próxima a la) uniforme? Calculemos $F_Y$, para $0\le a\le 100$:
\begin{equation*}
  F_Y(a)=\mathds{P}\left(X^2\le a\right)=\mathds{P}\left(X\le\sqrt{a}\right)=\frac{\sqrt{a}}{10}.
\end{equation*}
Así, la función de densidad de probabilidad $f_Y$ para el área es, para $0<y<199$ metros cuadrados %TODO usar el paquete siunitx
, dado por
\begin{equation*}
  f_Y(y)=\diff{F_Y(y)}{y}=\diff{\sqrt{y}}{y}=\frac{1}{20\sqrt{y}}.
\end{equation*}
Esto significa que los edificios con áreas pequeñas son %TODO
, porque $f_Y$ explota cerca de 0--vea también la %TODO singularidad
, en el cual %TODO
$f_Y$.

Sorpresivamente, esto no es muy visible en %TODO
, un ejemplo donde debemos crear en nuestros cálculos más que en nuestros ojos. En la figura las ubicaciones de los edificios son generados por el proceso de Poisson, el tema del capítulo 12.

Suponga que un %TODO
tiene que hacer una oferta en el precio de las %TODO
de los edificios. El monto concreto que necesitará será proporcional al área $X^2$ de un edificio. Así, su problema es: ¿cuál es el área esperada de un edificio? Con $f_Y$ de %TODO \eqref{7.1}
él encuentra
\begin{equation*}
  \mathds{E}\left[X^2\right]=\mathds{E}\left[Y\right]=\int_{0}^{199}\! y\cdot\frac{1}{20\sqrt{y}}\dl y=\int_{0}^{100}\frac{\sqrt{y}}{20}\dl y={\left[\frac{1}{20}\frac{2}{3}y^{\tfrac{3}{2}}\right]}_{0}^{100}=33\tfrac{1}{3}\mathrm{m}^2.
\end{equation*}
   %    ODO GRaficar con R la función dada.
La densidad de probabilidad del cuadrado de una variable aleatoria $U\left(0,10\right)$.

Es interesante notar que \emph{realmente} necesitamos hacer este cálculo, porque el valor esperado \emph{no} es simplemente el producto del ancho esperado y la profundidad esperada, que es % \SI{25}{\metre\per\square}

Sin embargo, existe una manera mucho más fácil en el cual el %TODO constructor
puede obtener este resultado. El podría tener el argumento que el valor del \emph{área} es $x^2$ cuando $x$ es el ancho, y que él debería tomar el peso  promedio de \emph{esos} valores, donde el peso de cada ancho $x$ es dado por el valor $f_X(x)$ de la densidad de probabilidad de $X$. Entonces él podría tener calculado
\begin{equation*}
  \mathds{E}\left[X^2\right]=\int_{-\infty}^{\infty}x^2f_X(x)\dl x=\int_{0}^{10}x^2\cdot\frac{1}{10}\dl x={\left[\frac{1}{30}x^3\right]}_{0}^{10}=33\tfrac{1}{3}%TODO.
\end{equation*}
Esto de hecho es un teorema matemático que esto es \emph{siempre} una manera correcta de calcular los valores esperados de funciones de variables aleatorias.

\begin{theorem}[La fórmula del cambio de variable]
  Sea $X$ una variable aleatoria, y sea $g\colon\mathds{R}\rightarrow\mathds{R}$ una función.

  Si $X$ es discreta, tomando los valores $a_1, a_2, \ldots$, entonces
  \begin{equation*}
    \mathds{E}%\left[g\left(\right]
  \end{equation*}
\end{theorem}

% TODO Agregar la definición de la distribución Gamma.

% TODO Justificar el hecho que la función exponencial se puede expresar como -1\lambda que tiene la distribución exponencial.

\subsection{Ejercicios}

\begin{enumerate}
	\item En cada uno de los siguientes ejemplos, intente indicar cuándo el proceso de Poisson sería un buen modelo.
	\begin{enumerate}
		\item La cantidad de bancarrotas de las empresas en los Estados Unidos.
		\item Las  veces que un pollo pone huevos.
		\item Las veces que un avión se estrella en un registro mundial.
		\item Las ubicaciones de las palabras deletreadas en un libro.
		\item Las cantidad de los accidentes de tránsito en una intersección.
	\end{enumerate}
	\item La cantidad de clientes que visitan un banco en un día es modelado por la distribución de Poisson. Se sabe que la probabilidad de que no haya clientes es de $0.00001$. ¿Cuál es el número esperado de clientes?
	\item Sea $N$ una variable aleatoria con distribución $\mathrm{Poiss}(4)$. ¿Cuánto es $\mathds{P}\left(N=4\right)$?
	\item Sea $X$ una variable aleatoria con distribución $\mathrm{Poiss}(2)$. ¿Cuánto es $\mathds{P}\left(X\le1\right)$?
	\item El número de errores de un disco duro es modelado como una variable aleatoria de Poisson con la esperanza de un error en cada Mb, esto es \SI{100}{\mega\byte\per\second}, esto es, en cada $2^{20}$ bytes.
	\begin{enumerate}
		\item ¿Cuál es la probabilidad de que haya al menos un error en un sector de $512$ bytes?
		\item El disco duro es una unidad de disco de \SI{18.62}{\giga\byte} con \num{39054015} sectores. ¿Cuál es la probabilidad de que haya al menos un error en el disco duro?
	\end{enumerate}
	\item Una cierta marca de alambre de cobre tiene fallas en cada \SI{40}{\centi\metre}. Modele las  ubicaciones de las fallas como un proceso de Poisson. En un intervalo de tiempo de \SI{20}{\minute} es dividido en ranuras de tiempo de \SI{10}{\second}. En un cierto punto a lo largo de la autopista, el número de autos que pasan es registrado para cada ranura de \SI{19}{\second}. Sea $n_j$ el número de ranuras en las que $j$ autos han pasado para $j=0,\ldots,9$. Suponga que uno encuentra
	\centering
	\begin{tabular}{c|cccccccccc}
		$j$ 	& $0$ & $1$ & $2$ & $3$ & $4$ & $5$ & $6$ & $7$ & $8$ & $9$ \\
		\hline
		$f(x)$& $19$ & $38$ & $28$ & $20$ & $7$ & $3$ & $4$ & $0$ & $0$ & $1$
	\end{tabular}\quad.

	Note que el número total de autos que pasan en esos $20$ minutos es $230$.
	\begin{enumerate}
		\item ¿Qué elegirías para el parámetro de intensidad $\lambda$?\label{12.7a}
		\item Suponga que estima la probabilidad de que cero autos pasando en una ranura de tiempo de \SI{10}{\second} por $n_0$ dividido por el número total de ranuras de tiempo. ¿Eso está de acuerdo (razonablemente) con el valor que sigue desde su respuesta en~\ref{12.7a}? %TODO dequeismo
		\item  ¿Qué tomarías para la probabilidad que $10$ autos pasan en una ranura de tiempo de \SI{10}{\second}?
	\end{enumerate}
	\item Sea $X$ una variable aleatoria de Poisson con parámetro $\mu$.
	\begin{enumerate}
		\item Calcule $\mathds{E}\left[X\left(X-1\right)\right]$.
		\item Calcule $\mathrm{Var}\left(X\right)$, usando el siguiente hecho:
		\begin{equation*}
		\mathrm{Var}\left(X\right)=\mathds{E}\left[X\left(X-1\right)\right]+\mathds{E}\left[X\right]-{\left(\mathds{E}\left[X\right]\right)}^2.
		\end{equation*}
	\end{enumerate}
	\item Sea $Y_1$ y $Y_2$ dos variables aleatorias de Poisson independientes con parámetros $\mu_1$ y $\mu_2$, respectivamente. Muestre que $Y=Y_1+Y_2$ también es una distribución de Poisson. En vez de usar la regla de adicción en la Sección 11.1 como en el ejercicio 11.2, puede probar esto sin ningún cálculo considerando el número de puntos de un proceso de Poisson (con intensidad $1$) en dos intervalos disjuntos de longitud $\mu_1$ y $\mu_2$.
	\item Sea $X$ una variable aleatorias con una distribución $\mathrm{Pois}\left(\mu\right)$. Muestre lo siguiente. Si $\mu<1$, entonces las probabilidades $\mathds{P}\left(X=k\right)$ son primero crecientes, luego decreciente (confer ). ¿Qué ocurre si $\mu=1$? %TODO Figura
	\item Considere el proceso de Poisson unidimensional con intensidad $\lambda$. Muestre que el número de puntos en $\left[0,t\right]$, \emph{dado} que el número de puntos en $\left[0,2t\right]$ es igual a $n$, tiene una distribución $\mathrm{Bin}\left(n,\tfrac{1}{2}\right)$.

	\emph{Ayuda:} escriba el evento $\left\{N\left(\left[0,s\right]\right)=k,N\left(\left[0,2s\right]\right)=n\right\}$ como la intersección de eventos (¡independientes!) $\left\{N\left(\left[0,s\right]\right)=k\right\}$ y $\left\{N\left(\left[0,2s\right]\right)=n-k\right\}$.
	\item Consideremos el proceso de Poisson unidimensional. Suponga que para algunos $\alpha>0$ se da que hay exactamente dos puntos $\left[0,a\right]$, o en otras palabras: $N_\alpha=2$. El objetivo de este ejercicio es determinar la distribución conjunta de $X_1$ y $X_2$, las ubicaciones de los dos puntos, sujeto a $N_\alpha=2$.
	\begin{enumerate}
		\item Pruebe que para $0<s<t<\alpha$\label{12.12a}
		\begin{multline*}
			\mathds{P}\left(X_1\le X_2\le t, N_\alpha=2\right) \\
			=\mathds{P}\left(X_2\le t, N_\alpha=2\right)-\mathds{P}\left(X_1>s,X_2\le t, N_\alpha=2\right).
		\end{multline*}
		\item Deduzca de~\ref{12.12a} que\label{12.12b}
		\begin{equation*}
			\mathds{P}\left(X_1\le s,X_2\le t,N_\alpha=2\right)=e^{-\lambda\alpha}\left(\frac{\lambda^2t^2}{2!}-\frac{\lambda^2{\left(t-s\right)}^2}{2!}\right).
		\end{equation*}
		\item Deduzca de~\ref{12.12b} que $0<s<t<\alpha$
		\begin{equation*}
			\mathds{P}\left(X_1\le s, X_2\le t\mid N_\alpha=2\right)=\frac{t^2-{\left(t-s\right)}^2}{\alpha^2}.
		\end{equation*}
	\end{enumerate}
	\item Al caminar por un prado encontramos dos tipos de flores, margaritas y dientes de león. Mientras caminamos en línea recta, modelamos las posiciones de las flores que encontramos con un proceso de Poisson con intensidad $\lambda$. Parece que aproximadamente una de cada cuatro flores es margarita. Olvidando los dientes de león, ¿cómo se ve el proceso de las \emph{margaritas}? Esta pregunta será respondida con los siguientes pasos:
	\begin{enumerate}
		\item Sea $N_t$ el número total de flores, $X_t$ el número de margaritas, e $Y_t$ el número de dientes de león que encontramos durante los primeros $t$ minutos de nuestro paseo. Note que $X_t+Y_t=N_t$. Suponga que cada flor es un margarita con probabilidad $1/4$, independiente de las otras flores. Argumenta que\label{12.13a}
		\begin{equation*}
			\mathds{P}\left(X_t=n,Y_t=m\mid N_t=n+m\right)=\binom{n+m}{n}{\left(\frac{1}{4}\right)}^n{\left(\frac{3}{4}\right)}^m.
		\end{equation*}
		\item Muestre que
		\begin{equation*}
			\mathds{P}\left(X_t=n,Y_t=m\right)=\frac{1}{n!}\frac{1}{m!}{\left(\frac{1}{4}\right)}^n{\left(\frac{3}{4}\right)}^me^{-\lambda t}{\lambda t}^{n+m},
		\end{equation*}
		condicionando sobre $N_t$ y usando~\ref{12.13a}.
		\item Escribiendo $e^{-\lambda t}=e^{-\left(\lambda/4\right)t}e^{-\left(3\lambda/4\right)t}$ y sumando sobre $m$, muestre que\label{12.13c}
		\begin{equation*}
			\mathds{P}\left(X_t=n\right)=\frac{1}{n!}e^{-\left(\lambda/4\right)t}{\left(\frac{\lambda t}{4}\right)}^n.
		\end{equation*}
		Ya que está claro que los números de margaritas que encontramos en intervalos de tiempo disjuntos son independientes, podríamos concluir de~\ref{12.13c} que el proceso $\left(X_t\right)$ es de nuevo un proceso de Poisson, con intensidad $\lambda/4$. Uno dice a menudo que el proceso $\left(X_t\right)$ es obtenido por \emph{adelgazamiento} del proceso $\left(N_t\right)$. En nuestro ejemplo este corresponde a recoger todos los dientes de león.
	\end{enumerate}
	\item En este ejercicio nos fijamos en un ejemplo simple de variables aleatorias $X_n$ que tiene la propiedad que sus distribuciones convergen a las distribuciones de una variable aleatorias $X$ cuando $n\to\infty$. Sea para $n=1,2,\ldots$ las variables aleatorias definidas por
	\begin{equation*}
	P\left(X_n=0\right)=1-\frac{1}{n}\quad\text{y}\quad\mathds{P}\left(X_n=7n\right)=\frac{1}{n}.
	\end{equation*}
	\begin{enumerate}
		\item Sea $X$ una variable aleatorias que es igual a $0$ con probabilidad $1$. Muestre que para todo $a$, las funciones de masa de probabilidad $p_{X_m}(a)$ de $X_n$ converge a la función de probabilidad de masa $p_X(a)$ de $X$ cuando $n\to\infty$. Note que $\mathds{E}\left[X\right]=0$.
		\item Muestre que sin embargo $\mathds{E}\left[X_n\right]=7$ para todo $n$.
	\end{enumerate}
\end{enumerate}
%%% Local Variables:
%%% mode: latex
%%% TeX-master: "../main"
%%% End:

%% --------------------
%% |   Bibliography   |
%% --------------------
%% Add entry to the table of contents for the bibliography

\vfill                                                                           
\nocite{*}                                                                       
\printbibliography[title={Referencias bibliográficas},heading=bibintoc]

\end{document}
https://stackoverflow.com/questions/2199463/emacs-auctex-run-command-on-file-that-is-not-currently-open

\begin{lem}[Alinhac-Lerner {\cite[p.~37]{a-l}}]
ith the output