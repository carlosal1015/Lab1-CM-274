\documentclass[12pt,a4paper]{kommaexam}

\begin{document}

\chapter{El proceso de Poisson}
\label{chap:1}

En varios fenómenos aleatorios encontramos, no solo uno o dos variables aleatorias que juegan un rol, sino una colección completa. En este caso frecuentemente habla de un \textit{proceso} aleatorio. El proceso de Poisson es un tipo simple de proceso aleatorio, que modela la ocurrencia de puntos aleatorios en el espacio o tiempo. Existen números formas en el que proceso de puntos aleatorios aumenta: algunos ejemplos son presentados en la primera sección. El proceso de Poisson describe en cierto sentido la \textit{forma más aleatoria} para distribuir puntos en el espacio o tiempo. Este es hecho más preciso con las nociones de homegeneidades y la independencia.

\section{Los puntos aleatorios}

Ejemplos típicos de la ocurrencia de puntos aleatorios son: tiempos de llegadas de mensajes de email en un servidor, las veces en que los asteroides golpean la tierra, tiempos de llegada de materiales radiactivos en un contador Geiger, veces en el que su computadora falla y los tiempos en que fallan los componentes electrónicos y los tiempos de llegada de personas a una bomba en un oasis.
% TODO Traducir pump correctamente.

Ejemplos de la ocurrencia de puntos aleatorios son: las ubicaciones de los impactos de asteroides con la tierra (dimensión $2$), las ubicaciones de las imperfecciones un material (dimensión $3$), y las ubicaciones de los árboles en un bosque. (dimensión $2$).

Algunos de estos fenómenos son mejor modelados por el proceso de Poisson que otros. Hablando libremente, uno podría decir que el proceso de Poisson frecuentemente aplica a situaciones donde existen una población muy grande, y cada miembro de la población tiene una pequeña probabilidad de producir un punto del proceso. Esto es, por ejemplo, bien cumplido en el ejemplo del contador Geiger donde, en una gran colección de átomos, solo unos pocos emitirán una partícula radiactiva. Una propiedad del proceso de Poisson, como veremos en breve, es que los puntos pueden estar arbitrariamente juntos. Por lo tanto, las ubicaciones de los árboles no están tan bien modeladas por el proceso de Poisson.

\end{document}
