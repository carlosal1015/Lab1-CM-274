% arara: clean: {extensions: ['log','aux','bbl','bcf','blg','run.xml','synctex.gz']}
% arara: xelatex: {shell: yes, synctex: yes}
% arara: xelatex: {shell: yes, synctex: yes}
% arara: biber
% arara: xelatex: {shell: yes, synctex: yes}
% arara: xelatex: {shell: yes, synctex: yes}
% arara: clean: {extensions: ['log','aux','bbl','bcf','blg','run.xml','synctex.gz']}
\RequirePackage{etex} % Increase the number of tokens up tp 2^15.
\documentclass[
	11pt,
	answers,
	addpoints,
	a4paper,
	tittlepage,
	DIV=15,
	headsepline,
	abstract=true
]{kommaexam}%,cancelspace
\usepackage{wallpaper}
\usepackage{amsthm}
\usepackage[libertine,scaled=1.1]{newtxmath}
\usepackage[lite]{mtpro2}
\usepackage[ISO]{diffcoeff}
\usepackage[makeroom]{cancel}
\usepackage[no-math]{fontspec}
\usepackage{libertine}
\usepackage[scaled=.92]{sourcesanspro}
\usepackage[scaled=.78]{beramono}
\usepackage[protrusion=true,expansion=false]{microtype} % IMPORTANT expansion = false for XeTeX engine!!
\usepackage{dsfont}
\usepackage{siunitx}
\sisetup{binary-units=true,per-mode=symbol}
\usepackage{graphicx}
\graphicspath{ {./images/} }
\usepackage{caption}
\usepackage[shortlabels]{enumitem}
\usepackage{ragged2e}

\usepackage[citestyle=numeric,style=numeric,backend=biber]{biblatex}

\DeclareSymbolFont{operators}{\encodingdefault}{\familydefault}{\mddefault}{n}
\DeclareMathAlphabet{\mathit}{\encodingdefault}{\familydefault}{\mddefault}{it}
\DeclareMathAlphabet{\mathbf}{\encodingdefault}{\familydefault}{\bfdefault}{n}

\let\oldref\ref
\renewcommand{\ref}[1]{(\oldref{#1})}

\addbibresource{./bibliography/bib.bib}

\renewcommand{\contentsname}{Tabla de contenidos}
\renewcommand{\listfigurename}{Lista de figuras}
\renewcommand{\figurename}{Figura}
\renewcommand{\abstractname}{Resumen}

\pagestyle{headandfoot}
\runningheadrule
\firstpageheader{CM--274 A}{Introducción a la Estadística y las Probabilidades}{2018--III}
\runningheader{CM--274 A}
{Introducción a la Estadística y las Probabilidades}
{2018--III}
\firstpageheadrule
\firstpagefootrule
\firstpagefooter{Facultad de Ciencias}{Página \thepage\ de \numpages.}{}
%\runningfooter{Facultad de Ciencias}{Página \thepage\ de \numpages.}{Grupo E}
\footer{FC--UNI}
{\iflastpage{Fin del proyecto}{Por favor continúe en la siguiente página\ldots}}
{Página \thepage\ de \numpages.}
\runningfootrule

\renewcommand{\solutiontitle}{\noindent\textbf{Solución}\par\noindent}

\definecolor{SolutionColor}{rgb}{0.2,0.9,1}
\colorsolutionboxes
\definecolor{SolutionBoxColor}{rgb}{0.2,0.9,1}

\vqword{Pregunta}
\vpword{Puntos}
\vsword{Puntaje}

\setkomafont{title}{\huge\sffamily\bfseries}
\setkomafont{subtitle}{\normalfont\large}

\newcommand{\theinstitute}{\\[3\baselineskip]\Large
	Escuela Profesional de Matemática\\
	de la Facultad de Ciencias\\Universidad Nacional de Ingeniería
}
\newcommand{\thethesistype}{}
\newcommand{\thereviewerone}{}
\newcommand{\thereviewertwo}{}
\newcommand{\theeditstart}{}
\newcommand{\theeditend}{}

%% formatting commands for titlepage
\newcommand{\thesistype}[1]{\subtitle{\vskip3em #1 por los alumnos}%
	\renewcommand{\thethesistype}{#1}}
\newcommand{\myinstitute}[1]{\renewcommand{\theinstitute}{#1}}
\newcommand{\reviewerone}[1]{\renewcommand{\thereviewerone}{#1}}
\newcommand{\reviewertwo}[1]{\renewcommand{\thereviewertwo}{#1}}

\newcommand{\editingtime}[2]{%
	\renewcommand{\theeditstart}{#1}%
	\renewcommand{\theeditend}{#2}%
	%% do not show the date
	\date{}
}

\newcommand{\settitle}{%
	\publishers{%
		\large
		\vskip4em
		\begin{tabular}{l l}
		Profesor de teoría: & \thereviewerone\\
		Profesor de Laboratorio: & \thereviewertwo
		\end{tabular}
		\theinstitute\\[4em]
		\vskip4em
		\theeditstart{} -- \theeditend
	}
}

\titlehead{%
	\centering\hspace*{-0.55cm}\includegraphics[height=2.5cm]{1}
	\ThisCenterWallPaper{1}{title-background.pdf}
}
\theoremstyle{definition}
\newtheorem{definition}{Definición}
\newtheorem{theorem}{Teorema}
\newtheorem{corollary}{Corolario}

\usepackage{xpatch}
\makeatletter
\xpatchcmd{\@thm}{\thm@headpunct{.}}{\thm@headpunct{}}{}{}
\makeatother

\author{
	Aznarán Laos Carlos Alonso	\quad\hfill 20162720C\\
	Janampa Bautista Edgar Joel		\quad\hfill 20151340J\\
	Paz Diaz Rodrigo Alonso	\quad\hfill 20141457A\\
}

\title{Segundo proyecto\\
	Introducción a la Estadística y las Probabilidades\\
	CM-274 A}

\thesistype{Informe matemático del grupo conformado}

%% The reviewers are the professors that grade your thesis
\reviewerone{Ph.D Ángel Enrique Ramírez Gutiérrez.}
\reviewertwo{Lic. José Fernando Zamudio Peves.}

\settitle
%% Please enter the start end end time of your thesis
\editingtime{25 de febrero del 2019}{4 de marzo del 2019}

\begin{document}

	\begin{coverpages}
		\clearpage\maketitle
		\thispagestyle{empty}
	\end{coverpages}

	\begin{abstract}
	Este trabajo.
\end{abstract}

	\tableofcontents

	\newpage

	\listoffigures

	\newpage

	\section{Introducción}
	

	\subsection{Consejos para erizos o constantes pueden variar}

	Este ensayo proporciona una breve introducción e historia del proceso de Poisson y consideramos un problema simple que ofrece entretenimiento analítico, junto con una prueba novedosa de resultado estándar.

	Es un comentario trivial pero verdadero decir que hay dos procesos aleatorios fundamentales, uno es el proceso de Poisson y el otro es el movimiento browniano. Cada uno de ellos es fundamental de dos maneras; En primer lugar, parecen describir muy bien gran parte del mundo natural, y en segundo lugar, se aconseja al estudiante que estudie y domine a estos dos antes de analizar procesos más duros y recónditos. Esta nota ofrece una breve visión de algunos aspectos del proceso de Poisson, que técnicamente es, con mucho, el más manejable de los dos.

	Primero recordamos una propiedad elemental de la distribución binomial:

	\[
		p\left(k\right)=\binom{n}{k}p^{k}{\left(1-p\right)}^{1-k}.\quad 0\le p\le 1,\quad o\le k\le n.
	\]

	Muy a menudo, en la vida real ocurre que $n$ es muy grande, y $p$ es muy pequeño, pero la media de la distribución, $\lambda=np$, está cerca de algún valor común (como $1$, o $e$ o $\sqrt{2}$ por ejemplo). Las aproximaciones a menudo son útiles en tales casos, y es un ejercicio de rutina mostrar que cuando $n\to\infty$, con $\lambda$ fijo, (de modo que $p=\lambda n^{-1}\to0$), tenemos
	\begin{equation}\label{eq:1}
		p\left(k\right)=\binom{n}{k}{\left(\frac{\lambda}{n}\right)}^{k}{\left(1-\frac{\lambda}{n}\right)}^{n-k}\to\frac{e^{-\lambda}\lambda^{k}}{k!},\quad k\ge 0.
	\end{equation}

	Esta es la conocida distribución de probabilidad de Poisson con el parámetro $\lambda$. Para una ilustración, suponga que tiene un gran bloque de material radiactivo, el bloque de granito que consta de Dartmoor, por ejemplo. El número de átomos activos es enorme, la posibilidad de que un átomo dado emita una partícula $\alpha$ es extremadamente pequeño, el número esperado de partículas$\alpha$ emitidas durante (por ejemplo) el lunes es moderado, y se aplica la aproximación de Poisson. Este y otros ejemplos similares sugieren otra propiedad crucial de tales procesos. Sea $N\left(s\right)$ el número de partículas emitidas durante el intervalo de tiempo $\left(0,s\right)$, de modo que $\tilde{N}=N\left(t\right)-N\left(s\right)$ es el número emitido durante $\left(s,t\right)$, donde $0<s<t$. Estas dos variables aleatorias $N\left(S\right)$ y $\tilde{N}$ obviamente se pueden asumir como independientes entre sí; empíricamente se encuentra que las partículas emitidas el lunes no afectan la cantidad emitida el martes. Estas consideraciones llevan a la siguiente definición:
	\begin{definition}
		El \textit{proceso de Poisson} es una colección de variables aleatorias de valores enteros no negativos $N\left(t\right), t\ge0$, tal que $N\left(0\right)=0$ y para todo positivo $s_{i}$ y $t_{i}$, con $s_{i}<t_{i}<s_{i+1}$, las variables aleatorias $\tilde{N}_{i}=N\left(t_{i}\right)-N\left(s_{i}\right)$, $1\le i\le n$ son mutuamente independientes, y $\tilde{N}_{i}$ tiene una distribución de Poisson con parámetro $\lambda\left(t_{i}-s_{i}\right)$. Escribimos
		\begin{equation}\label{eq:2}
		p_{n}\left(t\right)=\mathds{P}\left(N\left(t\right)=n\right)=\frac{e^{-\lambda t}{\left(\lambda t\right)}^{n}}{n!}
		\end{equation}
	\end{definition}
	Se dice que el proceso tiene incrementos independientes, y se dice que los incrementos son estacionarios porque la distribución de $N\left(t+s\right)-N\left(s\right)$ es la misma que la de $N\left(t\right)$.

	Es útil e importante ver que esta idea se utiliza para derivar las propiedades del proceso $N\left(t\right)$ de una manera diferente. La idea es que observemos lo que puede suceder con el proceso $N\left(t\right)$ durante un intervalo arbitrariamente pequeño $\left(t,t+\delta t\right)$. %TODO diff
	Por lo que hemos dicho anteriormente, el incremento $N\left(t+\delta t\right)-N\left(t\right)$ será independiente de los valores del proceso $N\left(s\right)$ para $0\le s\le t$. Además, para el problema que hemos examinado, podemos suponer que la probabilidad de que se emita una partícula $\alpha$ durante $\left(t,t+\delta t\right)$ es aproximadamente $\lambda\delta t$, la probabilidad de que no se emita ninguna es aproximadamente $1-\lambda\delta t$, y la probabilidad de dos o más es menor que $c{\left(\delta t\right)}^{2}$, para alguna constante $c$. Ahora, usando la probabilidad condicional y la independencia, podemos expandir la probabilidad $p_{n\left(t+\delta t\right)}$ de la siguiente manera.

	Para $n\ge 1$,
	\begin{align*}
		p_{n}\left(tt+\delta t\right)
		&=\mathds{P}\left(N\left(t\right)\right)=n\text{ y } \left(N\left(t+\delta t\right) - N\left(t\right)\right)=0\\
		&+\mathds{P}\left(N\left(t\right)=n-1\text{ y }\left(N\left(t+\delta t\right)-N\left(t\right)\right)=1\right)%TODO: phantom
		\\
		&+\mathds{P}\left(N\left(t\right)=n-r\text{ y }\left(N\left(t+\delta t\right)-N\left(t\right)\right)=r>1\right)\\
		&=p_{n}\left(t\right)\left(1-\lambda\delta t\right)+p_{n-1}\left(t\right)\lambda\delta t+r\left(t\right){\left(\delta t\right)}^{2}
	\end{align*}
	donde $r\left(t\right)\delta t\to0$ siempre que $\delta t\to0$. Dividiendo por $\delta t$, y permitiendo que $\delta t\to0$, nos da
	\begin{equation}\label{eq:3}
		p^{\prime}_{n}\left(t\right)=\diff{p_{n}\left(t\right)}{t}=-\lambda p_{n}\left(t\right)+\lambda p_{n-1}\left(t\right),\qquad n\ge1,
	\end{equation}
	y un argumento similar cuando $n=0$ nos da
	\begin{equation}\label{eq:4}
		p^{\prime}_{0}\left(t\right)=-\lambda p_{0}\left(t\right).
	\end{equation}
	Esto es fácil de verificar que la solución de \eqref{eq:3} y \eqref{eq:4} tal que $N\left(0\right)=0$ es provista por \eqref{eq:2}.

	Podemos inmediatamente deducir una propiedad importante de este proceso: las veces entre emisiones, (es decir, las veces entre instantes en el cual $N\left(t\right)$ incrementa una unidad) son variables aleatorias independientes exponencialmente distribuidas. Para ver esto, sea $X_{1}$ el time del primer salto. Así,

	\[
		N\left(t\right)=
		\begin{cases}
			0&,\quad t<X_{1}\\
			1&,\quad t=X_{1}.
		\end{cases}
	\]
	Entonces
	\begin{align*}
		\mathds{P}\left(X_{1}>t\right)
		&=\mathds{P}\left(N\left(t\right)=9\right)\\
		&=p_{0}\left(t\right)\\
		&=e^{-\lambda t}.
	\end{align*}

	Así, $X_{1}$ tiene un densidad exponencial con parámetro $\lambda$. El esto de nuestro reclamo sigue del hecho que $n\left(t\right)$ tiene incrementos estacionarios independientes.

	Completamos el circuito alrededor de estas caracterizaciones del proceso de Poisson por nada que si los eventos ocurren en una sucesión de veces $S_{n}=\sum_{r=1}^{n}X_{r}$, $n\ge1$, donde $X_{r}$ son variables aleatorias independientes y exponencialmente distribuidas, entonces se puede mostrar que el proceso $N\left(t\right)$ que cuenta el número de eventos en $\left[0,t\right]$ es de hecho un proceso de Poisson.

	Aquí está otra notable propiedad del proceso de Poisson, que necesitamos usar más después. Suponga que para $t\ge0$ el proceso de Poisson $N\left(t\right)$ con parámetro $\lambda$ cuenta los instantes en que los mosquitos pican tu cuello, y el proceso de Poisson $M\left(t\right)$ con parámetro $\mu$ cuenta los instantes en los que las avispas aterrizan en tu cerveza. Parece razonable suponer que $M\left(t\right)$ y $N\left(t\right)$ son independientes; ¿cómo podrían colaborar? Entonces este es el caso que el proceso $K\left(t\right)=M\left(t\right)+N\left(t\right)$ que registra ambos tipos de ocurrencias, es un proceso de Poisson con parámetros $\kappa=\lambda+\mu$. Para ver esto simplemente note primero que para cualquier $t$, $K\left(t\right)$ es una variable aleatoria de Poisson con parámetro $\left(\lambda+\mu\right)$, porque
	\begin{align*}
		\mathds{P}\left(K\left(t\right)=k\right)
		&=\sum_{n=0}^{k}\mathds{P}\left(M\left(t\right)+n=k, N\left(t\right)=n\right)\\
		&=\sum_{n=0}^{k}e^{-\mu t}\frac{{\left(\mu t\right)^{k-n}}}{\left(k-n\right)!}e^{-\lambda t}\frac{{\left(\lambda t\right)}^{n}}{n!}\\
		&=e^{-\left(\lambda+\mu\right)t}\left\{\left(\lambda+\mu\right)\right\}
	\end{align*}

	\subsection{Sobre la distribución de probabilidad de las partículas $\alpha$}\label{sec:note}

	Sea $\lambda\dl t$ la probabilidad que una partícula $\alpha$ golpee la pantalla en un intervalo de tiempo $\dl t$. Si los intervalos de tiempo bajo la consideración son pequeños comparados con el período de tiempo de la sustancia radiactiva, podemos asumir que $\lambda$ es independiente de $t$. Ahora, sea $W_n(t)$ la probabilidad que $n$ partículas $\alpha$ golpeen la pantalla en un intervalo de tiempo $t$, entonces la probabilidad que $(n+1)$ partículas golpeen la pantalla en un intervalo $t+\dl t$ es la suma de dos probabilidades. En primer lugar, las $n+1$ partículas $\alpha$ pueden golpear la pantalla en el intervalo $t$ y ninguno en el intervalo $\dl t$. La probabilidad de que esto pueda ocurrir es $(1-\lambda\dl t)W_{n+1}(t)$. En segundo lugar, $n$ partículas $\alpha$ podrían golpear la pantalla en el intervalo $t$ y otra partícula $\alpha$ en el intervalo $\dl t$; la probabilidad de que esto ocurra es $\lambda\dl t W_n(t)$. Por lo tanto
	\[
		W_{n+1}\left(t+\dl t\right)=\left(1-\lambda\dl t\right)W_{n+1}(t)+\lambda\dl t W_n\left(t\right).
	\]
	Procediendo al límite, tenemos
	\[
		\diff{W_{n+1}}{t}=\lambda\left(W_n-W_{n+1}\right).
	\]
	Poniendo $n=0,1,2,\ldots$ en una sucesión tenemos el sistema de ecuaciones:
	\begin{align*}
		\diff{W_0}{t}&=-\lambda W_0,\\
		\diff{W_1}{t}&=-\lambda\left(W_0-W_1\right),\\
		\diff{W_2}{t}&=-\lambda\left(W_1-W_2\right),\\
		\vdots\quad &=\quad\vdots\\
	\end{align*}
	que son exactamente de la misma forma que las que se producen en la teoría de transformaciones radiactivas, excepto que los períodos de tiempo de las transformaciones deberían suponerse que son todos iguales.

	Las ecuaciones podrían ser resueltos multiplicando cada uno de ellos por $e^{\lambda t}$ e integrando. Dado que $W_0(0)=1$ y $W_n(0)=0$, tenemos una sucesión:
	\begin{align*}
		&& W_0&= e^{-\lambda t},\\
		\diff{W_1e^{\lambda t}}{t}&=\lambda, & W_1&= e^{-\lambda t},\\
		\diff{W_2e^{\lambda t}}{t}&=\lambda^2t, & W_2&=\frac{{\left(\lambda t\right)}^2}{2!}e^{-\lambda t},\\
	\end{align*}
	y así sucesivamente. Finalmente, tenemos
	\[
		W_n=\frac{{\left(\lambda t\right)}^n}{n!}e^{\lambda t}.
	\]
	El \emph{promedio} del número de  partículas $\alpha$ que golpean la pantalla en el intervalo $t$ es $\lambda t$. Poniendo esto igual a $x$, vemos que la probabilidad de que $n$ partículas $\alpha$ golpeen la pantalla en este intervalo es
	\[
		W_n=\frac{x^n}{n!}e^{-x}.
	\]

	El caso particular en el que $n=0$ se conoce desde hace tiempo. % TODO Chequear el libro Whitworth Choice and Chance 4th ed prop 51.

	Si usamos la analogía de arriba con la transformación radiactiva, el teorema simplemente nos dice que la cantidad de sustancia primaria restante después de un intervalo de tiempo $t$ es $e^{-\lambda t}$ si una cantidad unitaria estaba presente en el comienzo.

	El número \emph{probable} de partículas $\alpha$ que golpean la pantalla en un intervalo dado es
	\[
		p=\sum_{n=1}^{\infty}nW_n=xe^{-x}\sum_{n=1}^{\infty}\frac{x^{n-1}}{\left(n-1\right)!}=x.
	\]
	El número \emph{más probable} es obtenido por la búsqueda del valor máximo de $W_n$.

	Ya que $\frac{W_n}{W_{n-1}}=\frac{x}{n}$, este radio será mayor que $1$ siempre que $n<x$. Por lo tanto, si $n\lessgtr x$,
	\[
		W_n\lessgtr W_{n-1}
	\]
	si $n=x$, $W_n=W_{n-1}$. El valor más probable de $n$ es por lo tanto el entero siguiente mayor que $x$; si, sin embargo, $x$ es un entero, los números $x-1$ y $x$ son igualmente probable, y más probable que todos los otros.

	El valor de $\lambda$ que es calculado por el conteo del número total de partículas $\alpha$ que chocan la pantalla en un intervalo largo de tiempo $T$, no será generalmente el valor verdadero de $\lambda$. La desviación media desde el valor de verdad de $\lambda$ es calculado por la búsqueda de la desviación media del número total $N$ de partículas $\alpha$ observadas en el tiempo $T$ desde el número verdadero promedio $\lambda T$. Esta desviación media $D$ (error promedio) es, de acuerdo a la definición de Bessel y Gau\ss, la raíz cuadrada del valor probable del cuadrado de la diferencia $N-\lambda T$, y así es obtenido por las series % TODO Footnote de Bessel y Gauss.
	\begin{align*}
		D^2
		&=\sum_{n=0}^{\infty}\left(N-\lambda T\right)^2\frac{{\left(\lambda T\right)}^N}{N!}e^{-\lambda t}\\
		&=e^{-\lambda t}\sum_{N=0}^{\infty}\left[\frac{{\left(\lambda T\right)}^N}{\left(N-2\right)!}
		+\frac{{\left(\lambda T\right)}^{N+1}}{\left(N-1\right)!}+\frac{{\left(\lambda T\right)}^{N+2}}{\left(N\right)!}\right]
		=\lambda T.
	\end{align*}
	Así, $D=\sqrt{\lambda T}$, y la desviación media desde el valor de $\lambda$ es en consecuencia
	\[
		\frac{D}{T}=\sqrt{\frac{\lambda}{T}};
	\]
	esto es, varía inversamente como la raíz cuadrada de la longitud del intervalo de tiempo. Este resultado es de la misma forma que el clásico utilizado por E. v. Schweidler en el artículo  mencionado anteriormente.

	El valor probable de $|N-\lambda T|$ (error promedio) es mucho más difícil de calcular.% TODO citar Schwediler

	
\section{Definiciones}
\label{sec:defi}

\subsection{Distribución uniforme}

\begin{definition}
  Una variable aleatoria continua tiene \textit{distribución uniforme} en el intervalo $\left[a,b\right]$ si su función de densidad de probabilidad $f$ es dado por $f(x)=0$ si $x$ no está en $\left[a,b\right]$ y
  \begin{equation*}
    f(x)=\frac{1}{\beta-\alpha}\quad\text{para }\alpha \le x\le \beta.
  \end{equation*}
 Denotamos esta distribución por $U\left(\alpha,\beta\right)$.
\end{definition}

\subsection{La fórmula del cambio de variable}
\label{sec:change}

Con frecuencia no queremos calcular el valor esperado de una variable aleatoria $X$, pero %TODO
de una función $X$, como, por ejemplo, $X^2$. Entonces necesitamos determinar la distribución $Y=X^2$, por ejemplo para calcular la función de distribución $F_Y$ de $Y$ (este es un ejemplo del problema general de cómo las distribuciones bajo las transformaciones -- este tópico es es tema del capitulo 8). Para un ejemplo concreto, suponga un arquitecto quiere maximizar %TODO  Page 94
en el tamaño de los edificios: estos deben ser el mismo ancho y profunidad $X$, pero $X$ es uniformemente distribuido entre $0$ y $10$ metros. ¿Cuál es la distribución del área $X^2$ de un edificio? En particular, ¿será esta distribución (cualquier próxima a la) uniforme? Calculemos $F_Y$, para $0\le a\le 100$:
\begin{equation*}
  F_Y(a)=\mathds{P}\left(X^2\le a\right)=\mathds{P}\left(X\le\sqrt{a}\right)=\frac{\sqrt{a}}{10}.
\end{equation*}
Así, la función de densidad de probabilidad $f_Y$ para el área es, para $0<y<199$ metros cuadrados %TODO usar el paquete siunitx
, dado por
\begin{equation*}
  f_Y(y)=\diff{F_Y(y)}{y}=\diff{\sqrt{y}}{y}=\frac{1}{20\sqrt{y}}.
\end{equation*}
Esto significa que los edificios con áreas pequeñas son %TODO
, porque $f_Y$ explota cerca de 0--vea también la %TODO singularidad
, en el cual %TODO
$f_Y$.

Sorpresivamente, esto no es muy visible en %TODO
, un ejemplo donde debemos crear en nuestros cálculos más que en nuestros ojos. En la figura las ubicaciones de los edificios son generados por el proceso de Poisson, el tema del capítulo 12.

Suponga que un %TODO
tiene que hacer una oferta en el precio de las %TODO
de los edificios. El monto concreto que necesitará será proporcional al área $X^2$ de un edificio. Así, su problema es: ¿cuál es el área esperada de un edificio? Con $f_Y$ de %TODO \eqref{7.1}
él encuentra
\begin{equation*}
  \mathds{E}\left[X^2\right]=\mathds{E}\left[Y\right]=\int_{0}^{199}\! y\cdot\frac{1}{20\sqrt{y}}\dl y=\int_{0}^{100}\frac{\sqrt{y}}{20}\dl y={\left[\frac{1}{20}\frac{2}{3}y^{\tfrac{3}{2}}\right]}_{0}^{100}=33\tfrac{1}{3}\mathrm{m}^2.
\end{equation*}
   %    ODO GRaficar con R la función dada.
La densidad de probabilidad del cuadrado de una variable aleatoria $U\left(0,10\right)$.

Es interesante notar que \emph{realmente} necesitamos hacer este cálculo, porque el valor esperado \emph{no} es simplemente el producto del ancho esperado y la profundidad esperada, que es % \SI{25}{\metre\per\square}

Sin embargo, existe una manera mucho más fácil en el cual el %TODO constructor
puede obtener este resultado. El podría tener el argumento que el valor del \emph{área} es $x^2$ cuando $x$ es el ancho, y que él debería tomar el peso  promedio de \emph{esos} valores, donde el peso de cada ancho $x$ es dado por el valor $f_X(x)$ de la densidad de probabilidad de $X$. Entonces él podría tener calculado
\begin{equation*}
  \mathds{E}\left[X^2\right]=\int_{-\infty}^{\infty}x^2f_X(x)\dl x=\int_{0}^{10}x^2\cdot\frac{1}{10}\dl x={\left[\frac{1}{30}x^3\right]}_{0}^{10}=33\tfrac{1}{3}%TODO.
\end{equation*}
Esto de hecho es un teorema matemático que esto es \emph{siempre} una manera correcta de calcular los valores esperados de funciones de variables aleatorias.

\begin{theorem}[La fórmula del cambio de variable]
  Sea $X$ una variable aleatoria, y sea $g\colon\mathds{R}\rightarrow\mathds{R}$ una función.

  Si $X$ es discreta, tomando los valores $a_1, a_2, \ldots$, entonces
  \begin{equation*}
    \mathds{E}%\left[g\left(\right]
  \end{equation*}
\end{theorem}

	\section{El proceso de Poisson}\label{sec:12}

	En varios fenómenos aleatorios encontramos, no solo uno o dos variables aleatorias que juegan un rol, sino una colección completa. En este caso frecuentemente habla de un \textit{proceso} aleatorio. El proceso de Poisson es un tipo simple de proceso aleatorio, que modela la ocurrencia de puntos aleatorios en el espacio o tiempo. Existen números formas en el que proceso de puntos aleatorios aumenta: algunos ejemplos son presentados en la primera sección. El proceso de Poisson describe en cierto sentido la \textit{forma más aleatoria} para distribuir puntos en el espacio o tiempo. Este es hecho más preciso con las nociones de homogeneidad e independencia.

\subsection{Los puntos aleatorios}

	Ejemplos típicos de la ocurrencia de puntos aleatorios son: tiempos de llegadas de mensajes de correo electrónico en un servidor, las veces en que los asteroides golpean la tierra, tiempos de llegada de materiales radiactivos en un contador Geiger\footnote{Un contador Geiger es un instrumento que permite medir la radiactividad de un objeto o lugar.}, veces en el que su computadora falla y los tiempos en que fallan los componentes electrónicos y los tiempos de llegada de personas a una bomba de agua en un oasis.

	Ejemplos de la ocurrencia de los puntos aleatorios son: las ubicaciones de los impactos de asteroides con la tierra (dimensión $2$), las ubicaciones de las imperfecciones un material (dimensión $3$), y las ubicaciones de los árboles en un bosque. (dimensión $2$).

	Algunos de estos fenómenos son mejor modelados por el proceso de Poisson que por otros. De forma aproximada, uno podría decir que el proceso de Poisson se aplica con frecuentemente a situaciones donde existe una población muy grande, y cada miembro de este tiene una pequeña probabilidad de producir un punto del proceso. Esto es, por ejemplo, bien satisfecho en el ejemplo del contador Geiger donde, en una descomunal colección de átomos, solo unos pocos emitirán una partícula radiactiva. Una propiedad del proceso de Poisson--como veremos en breve--es que los puntos pueden estar arbitrariamente juntos. Por lo tanto, las ubicaciones de los árboles no están tan bien modeladas por el proceso de Poisson.

\subsubsection{Echando un vistazo más de cerca las llegadas aleatorias}

	Un ejemplo bien conocido que es usualmente modelado por el proceso de Poisson es que las llamadas que llegan a una central telefónica--la central está conectada a un gran número de personas que hacen llamadas telefónicas de vez en cuando. Este será nuestro ejemplo principal en esta sección.

	\noindent Las llamadas telefónicas llegan en tiempos aleatorios $X_{1}, X_{2}, \ldots$ a la central telefónica durante el tiempo de tiempo $\left[0,t\right]$.

	\begin{figure}[!ht]
		\centering
		\includegraphics[width=0.7\paperwidth]{img12_1a}
		\caption*{}
	\end{figure}

	\noindent Los dos supuestos básicos que hacemos en estas llamadas aleatorias son
	\begin{description}
		\item[Homogeneidad] La tasa $\lambda$ a la que se producen las llamadas es constante a través del tiempo: en un subintervalo de longitud $u$ la esperanza del número de llamadas telefónicas es $\lambda u$.
		\item[Independencia] El número de llegadas en intervalos de tiempo disjuntos son variables aleatorias independientes.
	\end{description}

	La homogeneidad es también llamado \textit{estacionario débil}. Denotamos el número total de llamadas en un intervalo $I$ por $N\left(I\right)$, abreviando $N\left(\left[0,t\right]\right)$ por $N_{t}$. Entonces, la homogeneidad implica que requerimos
	\[
		\mathds{E}\left[N_{t}\right]=\lambda t.
	\]

	\begin{figure}[!ht]
		\centering
		\includegraphics[width=0.7\paperwidth]{img12_1b}
		\caption*{}
	\end{figure}

	\noindent Para obtener la distribución de $N_{t}$, dividimos el intervalo $\left[0,t\right]$ en $n$ intervalos de longitud $\tfrac{t}{n}$. Cuando $n$ es lo suficientemente grande, cada intervalo $I_{j,n}=\left(\tfrac{\left(j-1\right)t}{n},\tfrac{jt}{n}\right]$ contendrá $0$ o $1$ llegada: Para un $n$ tan grande (que también satisface $n>\lambda t$), sea $R_{j}$ el número de llegadas en el intervalo de tiempo $I_{j,n}$. Como $R_{j}$ es $0$0 o $1$, $R_{j}$ tiene una distribución $\mathrm{Ber}\left(p_{j}\right)$ para algunos $p_{j}$. Recuerde que para una variable aleatoria de Bernoulli, $\mathds{E}\left[R_{j}\right]=0\cdot\left(1-p_{j}\right)+1\cdot p_{j}=p_{j}$. Por el supuesto de homogeneidad, para cada $j$
	\[
		p_{j}=\lambda\cdot\text{longitud de }I_{j,n}=\frac{\lambda t}{n}.
	\]
	Sumando el número de llamadas en los intervalos nos da el número total de llamadas, así
	\[
		N_{t}=R_{1}+R_{2}+\cdots+R_{n}.
	\]
	Por el supuesto de independencia, los $R_{j}$ son variables aleatorias independientes, por lo tanto, $N_{t}$ tiene una distribución $\mathrm{Bin}\left(n,p\right)$ con $p=\tfrac{\lambda t}{n}$.

	\begin{definition}
		Una variable aleatoria discreta $X$ tiene una \textit{distribución de Poisson} con parámetro $\mu$, donde $\mu>0$ si su función de masa de probabilidad $p$ es dado por
	\[
		p\left(k\right)=\mathds{P}\left(X=k\right)=\frac{\mu^{k}e^{-\mu}}{k!}\quad\text{para }k=0,1,2,\ldots.
	\]
	Denotamos esta distribución por $\mathrm{Pois}\left(\mu\right)$.
	\end{definition}

	\begin{figure}[!ht]
		\centering
		\includegraphics[width=0.7\paperwidth]{img12_1}
		\caption*{Las funciones de masa de probabilidad de $\mathrm{Pois}\left(\num{0.9}\right)$ y las distribución $\mathrm{Pois}\left(\num{5}\right)$.}
	\end{figure}

	\begin{theorem}[\textsc{La esperanza y la varianza de una distribución de Poisson}]
		Sea $X$ una variable aleatoria con distribución de Poisson con parámetro $\mu$, entonces
		\[
			\mathds{E}\left[X\right]=\mu\quad\text{y}\quad\mathrm{Var}\left(X\right)=\mu.
		\]
	\end{theorem}

\subsection{El proceso de Poisson unidimensional}

	Derivaremos algunas propiedades de la sucesión de puntos aleatorios $X_{1},X_{2},\ldots$ que hemos considerado en la sección anterior. Lo que derivamos hasta ahora es que para cualquier intervalo $\left(s,s+t\right]$ el número $N\left(\left(s,s+t\right]\right)$ de puntos $X_{i}$ en ese intervalo es una variable aleatoria con una distribución $\mathrm{Pois}\left(\lambda t\right)$.

\subsubsection{Tiempos de llegada}

	Las diferencias
	\[
		T_{i}=X_{i}-X_{i-1}
	\]
	se llaman tiempos de llegada. Aquí definimos $T_{1}=X_{1}$, el tiempo de la primera llegada. Para determinar la distribución de probabilidad de $T_{1}$, observamos que el evento $\left\{T_{1}>t\right\}$ que la primera llamada de llegada después del tiempo $t$ es el mismo que el evento $\left\{N_{t}=0\right\}$ que no se ha realizado ninguna llamada en $\left[0,t\right]$. Pero esto implica que
	\[
		\mathds{P}\left(T_{1}\le t\right)=1-\mathds{P}\left(T_{1}>t\right)=1-\mathds{P}\left(N_{t}=0\right)=1-e^{-\lambda t}.
	\]
	Por lo tanto, $T_{1}$ tiene una distribución exponencial con parámetro $\lambda$.

	Para calcular la distribución conjunta de $T_{1}$ y $T_{2}$, consideramos la probabilidad condicional de que $T_{2}>t$, dado que $T_{1}=s$, y usamos la propiedad de que las llegadas en intervalos diferentes son independientes:
	\begin{align*}
		\mathds{P}\left(T_{2}>t\mid T_{1}=s\right)
		&=\mathds{P}\left(\text{ninguna llegada en }\left(s,s+t\right]\mid T_{1}=s\right)\\
		&=\mathds{P}\left(\text{ninguna llegada en }\left(s,s+t\right]\right)\\
		&=\mathds{P}\left(N\left(\left(s,s+t\right]\right)=0\right)=e^{-\lambda t}.
	\end{align*}
	Como esta respuesta no depende de $s$, concluimos que $T_{1}$ y $T_{2}$ son independientes, y
	\[
		\mathds{P}\left(T_{2}>t\right)=e^{-\lambda t},
	\]
	es decir, $T_{2}$ también es una distribución exponencial con parámetro $\lambda$. En realidad, aunque la conclusión es correcta, el método para derivarlo no lo es, porque condicionamos el evento $\left\{T_{1}=s\right\}$ que tiene probabilidad cero, Este problema puede evitarse condicionando el hecho de que el evento $T_{1}$ se encuentra en un pequeño intervalo, pero eso no se hará aquí. Análogamente, se puede mostrar que los $T_{i}$ son independientes y que tienen una distribución $\mathrm{Exp}\left(\lambda\right)$.. Esta bonita propiedad permite dar una definición simple del proceso de Poisson unidimensional.
	\begin{definition}
		El \textit{proceso de Poisson} unidimensional con intensidad $\lambda$ es una sucesión de variables aleatorias $X_{1},X_{2},X_{3},\ldots$ son variables aleatorias independientes, cada una con una distribución de $\mathrm{Exp}\left(\lambda\right)$.
	\end{definition}
	Tenga en cuenta que la conexión con $N_{t}$ es la siguiente: $N_{t}$ es igual al número de $X_{1}$ que son menores (o igual) que $t$.
	\begin{definition}[\textsc{Los puntos del proceso de Poisson}]
		Para $i=1,2,\ldots$ la variable aleatoria $X_{i}$ tiene una distribución $\mathrm{Gam}\left(i,\lambda\right)$.
	\end{definition}

\subsubsection{La distribución de los puntos}

	Otra interesante pregunta es: si sabemos que $n$ puntos se generan en un intervalo, ¿dónde se encuentran estos puntos? Dado que la distribución del número de puntos solo depende de la longitud del intervalo, y no de su ubicación, basta con determinar esto para un intervalo que comience en $0$. Sea este $\left[0,a\right]$ un intervalo. Comenzamos con el caso más simple, donde existe un solo punto en $\left[0,a\right]$: suponga que $N\left(\left[0,a\right]\right)=1$. Luego, para $0<s<a$:
	\begin{align*}
		\mathds{P}\left(X_{1}\le s\mid N\left(\left[0,a\right]\right)=1\right)
		&=\frac{\mathds{P}\left(X_{1}\le s,N\left(\left[0,a\right]\right)=1\right)}{\mathds{P}\left(N\left(\left[0,a\right]\right)=1\right)}\\
		&=\frac{\mathds{P}\left(N\left(\left[0,s\right]\right)=1,N\left(\left(s,a\right]\right)=0\right)}{\mathds{P}\left(N\left(\left[0,a\right]\right)=1\right)}\\
		&=\frac{\lambda se^{-\lambda s}e^{-\lambda\left(a-s\right)}}{\lambda ae^{-\lambda a}}\\
		&=\frac{s}{a}.
	\end{align*}
	Encontramos que condicionar el evento $\left\{N\left(\left[0,a\right]\right)=1\right\}$, la variable aleatoria $X_{1}$ está uniformemente distribuida sobre el intervalo $\left[0,a\right]$. Supongamos ahora que hay dos puntos en $\left[0,a\right]\colon N\left(\left[0,a\right]\right)=2$. De una manera similar a lo que hicimos para un punto, podemos mostrar que
	\[
		\mathds{P}\left(X_{1}\le s, X_{2}\le t\mid N\left(\left[0,a\right]=2\right)\right)=\frac{t^{2}-{\left(t-s\right)}^{2}}{a^{2}}.
	\]
	Ahora, recuerde el resultado del ejercicio%TODO 9.17
	si $U_{1}$ y $U_{2}$ son dos variables aleatorias independientes, ambas distribuidas uniformemente sobre $\left[0,a\right]$, entonces la función de distribución conjunta de $V=\min\left(U_{1},U_{2}\right)$ y $Z=\max\left(U_{1},U_{2}\right)$ está dado por
	\[
		\mathds{P}\left(V\le s, Z\le t\right)=\frac{t^{2}-{\left(t-s\right)}^{2}}{a^{2}}\quad\text{para }0\le s\le t\le a.
	\]
	Por lo tanto, hemos encontrado que, si olvidamos su orden, los dos puntos $\left[0,a\right]$ son independientes y se distribuyen uniformemente sobre $\left[0,a\right]$. Con algo más de trabajo, esto se generaliza a un número arbitrario de puntos, y llegamos a la siguiente propiedad:
	\begin{definition}[\textsc{Ubicación de los puntos, dado su número}]
		Dado que el proceso de Poisson tiene $n$ puntos en el intervalo $\left[a,b\right]$, las ubicaciones de estos puntos son independientemente distribuidas, cada una con una distribución uniforme en $\left[a,b\right]$.
	\end{definition}

\subsection{Procesos de Poisson multidimensionales}

	Nuestra definición del procesos de Poisson unidimensional, que comienza con los tiempos de llegadas, no se generaliza fácilmente, porque se basa en el ordenamiento de los números reales. Sin embargo, podemos ampliar fácilmente los supuestos de independencia, homogeneidad y la propiedad de la distribución de Poisson. Para hacer esto necesitamos una versión de mayor dimensión del concepto de longitud. Denotamos el volumen $k$--dimensional de un conjunto $A$ en un espacio $k$--dimensional por $\mathrm{m}\left(A\right)$ es el volumen de $A$.
	\begin{definition}
		El \textit{proceso de Poisson} $k$--dimensional con intensidad $\lambda$ es una colección de puntos aleatorios $X_{1}, X_{2}, X_{3},\ldots$ que tienen la propiedad que si $N\left(A\right)$ denota el número de puntos en el conjunto $A$, entonces
		\begin{description}
			\item[Homogeneidad] La variable aleatoria $N\left(A\right)$ tiene una distribución de Poisson con parámetro $\lambda\mathrm{m}\left(A\right)$.
			\item[Independencia] Para conjuntos disjuntos $A_{1}, A_{2},\ldots, A_{n}$ las variables aleatorias $N\left(A_{1},\right),N\left(A_{2},\right),\ldots,N\left(A_{n},\right)$ son independientes.
		\end{description}
	\end{definition}
	En la Figura~\ref{fig:7.4}, las ubicaciones de los edificios que el arquitecto quería distribuir en un terreno de \num{100}$\times$\SI{300}{\metre} se generaron mediante un proceso de Poisson bidimensional. Esto se ha hecho de la siguiente manera, Se puede volver a mostrar que dado el número total de puntos en un conjunto, estos puntos se distribuyen uniformemente sobre el conjunto $A$. Esto lleva al siguiente procedimiento: primero se genera un valor $n$ a partir de una distribución de Poisson con el parámetro apropiado $\left(\lambda\cdot\text{área}\right)$, entonces uno genera $n$ veces un punto distribuido uniformemente sobre el rectángulo de \num{100}$\times$\SI{300}{\metre}.

	En realidad, se puede generar un proceso de Poisson de mayor dimensión de una manera muy similar a la forma natural en que se puede hacer para el proceso de una dimensión. Directamente a partir de la definición del proceso unidimensional, vemos que puede obtenerse generando puntos consecutivamente con ``huecos'' distribuidos exponencialmente. Explicaremos un procedimiento similar para la dimensión dos. Para $s>0$, sea
	\[
		M_{s}=N\left(C_{s}\right),
	\]
	donde $C_{s}$ es la región circular de radio $s$, centrado en el origen. Como la área de $C_{s}$ es $\pi s^{2}$. Sea $R_{i}$ la distancia del $i$--ésimo punto más cerca al origen. Esto es ilustrado en la Figura~\ref{fig:12.2}


	Note que $R_{i}$ es el análogo al $i$--ésimo tiempo de llegada para el proceso de Poisson unidimensional: tenemos de hecho que
	\[
		R_{i}\le s\qquad\text{si y solo si}\qquad M_{s}\le i.
	\]
	En particular, con $i=1$ y $s=\sqrt{t}$,
	\[
		\mathds{P}\left(_{1}^{2}\le t\right)=\mathds{P}\left(R_{1\le\sqrt{t}}\right)=\mathds{P}\left(M_{\sqrt{t}}\le i\right).
	\]

	\begin{figure}[!ht]
		\centering
		\includegraphics[width=0.7\paperwidth]{img12_2}
		\caption{El proceso de Poisson en el plano con dos regiones circulares de los dos puntos más cercanos al origen.}
		\label{fig:12.2}
	\end{figure}

	Así,
	\[
		\mathds{P}\left(R_{i}^{2}\le t\right)=1-e^{-\lambda\pi t}\sum_{j=0}^{i-1}\frac{{\left(\lambda\pi t\right)}^{j}}{j!},
	\]
	lo que significa que $R_{i}^{2}$ tiene una distribución $\mathrm{Gam}\left(i,\lambda\pi\right)$--como se vio en la página 157 %TODO \pageref.
	Ya que las distribuciones gamma surgen como sumas de distribuciones exponenciales independientes, podemos también escribir
	\[
		R_{i}^{2}=R_{i-1}^{2}+T_{i},
	\]
	donde los $T_{i}$ son variables aleatorias independientes de $\mathrm{Exp}\left(\lambda\pi\right)$ (y donde $R_{0}=0$). Tenga en cuenta que esto es bastante similar al caso unidimensional. Para similar el proceso de Poisson bidimensional a partir de una sucesión de las variables aleatorias independientes $U_{1},U_{2},\ldots$ de $U\left(0,1\right)$, por lo tanto, se puede proceder de la siguiente manera

\newpage

	\begin{center}
		\fbox{\fbox{\parbox{5.5in}{\centering
			Responda las preguntas en los espacios provistos en las hojas de preguntas. Si se queda sin espacio para una respuesta, continúe en la parte posterior de la página.
		}}}
	\end{center}

	\begin{questions}

		\question

	En cada uno de los siguientes ejemplos, intente indicar cuándo el proceso de Poisson sería un buen modelo.
	\begin{parts}
		\part La cantidad de bancarrotas de las empresas en los Estados Unidos.
		\part Las  veces que un pollo pone huevos.
		\part Las veces que un avión se estrella en un registro mundial.
		\part Las ubicaciones de las palabras deletreadas en un libro.
		\part Las cantidad de los accidentes de tránsito en una intersección.
	\end{parts}

	\begin{solutionordottedlines}
		\begin{parts}
			\part No, la cantidad de bancarrotas se dan en grupos.
			\part No, después de un huevo el pollo tarda un tiempo en producir otro.
			\part Es un buen candidato.
			\part Es otro buen candidato.
			\part Podría ser modelado por el proceso de Poisson si el cruce no es peligroso, sino las autoridades podrían tomar medidas y destruir la homogeneidad.
		\end{parts}
	\end{solutionordottedlines}

		\question%[15]
	Un sistema satelital consta de $n$ componentes y funciona en un día cualquiera si al menos $k$ de los $n$ componentes funcionan ese día. En un día lluvioso, cada uno de los componentes funciona independientemente con probabilidad $p_1$, mientras que en un día seco cada uno funciona independientemente con probabilidad $p_2$. Si la probabilidad de que llueva mañana es de $\alpha$, ¿cual es la probabilidad de que el sistema satelital funcione?
	\begin{solutionorlines}
		Hola
		\includegraphics[width=7cm]{example-image-b}
		\centering
		\captionof{figure}{This is a lovely figure}
		\label{fig:2}
		\justifying
		Hola
	\end{solutionorlines}

		\question%[10]
	Supuesto que el número de accidentes que ocurren en una autopista cada día es una variable aleatoria de \emph{Poisson} con parámetro $\lambda=3$.
	\begin{parts}
		\part Calcular la probabilidad que $3$ o más accidentes ocurran hoy.\label{part:a}
		\part Responder (\ref{part:a}) bajo el supuesto que al menos un accidente ocurrió hoy.
	\end{parts}

	\begin{solutionordottedlines}
		\begin{parts}
			\part Nos piden calcular $\sum_{i=3}^{\infty}p\left(x=i;\lambda t=3\right)$, donde $p\left(x;\lambda t\right)$ es la distribución de probabilidad de la variable aleatoria $X$ de Poisson, que representa el número de resultados que ocurren en un intervalo de tiempo por $t$,
			\begin{align*}
			\mathds{P}\left[X\ge 3\right]
			&= 1 - \mathds{P}\left[X< 3\right]\\
			&= 1 - \sum_{x=0}^{2}p\left(x;3\right)\\
			&= 0.57681
			\end{align*}
			\part Nos piden calcular $\mathds{P}\left[X\ge 3 \mid x\ge 1\right]$.
			\begin{align*}
			\mathds{P}\left[X\ge 3 \mid x\ge 1\right]
			&=\frac{\mathds{P}\left[X\ge 3 \cap x\ge 1\right]}{\mathds{P}\left[X\ge 1\right]}\\
			&=\frac{\mathds{P}\left[X\ge 3\right]}{1-\mathds{P}\left[X=0\right]}\\
			&=0.607.
			\end{align*}
		\end{parts}
	\end{solutionordottedlines}

			\question%[10]
		Supuesto que el número de ventas que ocurren en un tiempo específico es una variable aleatoria de Poisson con parámetro $\lambda$. Si cada evento es contado con probabilidad $p$, independiente un evento de otro evento, mostrar que el numero de eventos que son contados es una V.A de Poisson con parámetro $\lambda p$. También, dar un argumento intuitivo de porqué esto debería ser así. Como una aplicación del párrafo anterior, suponer que el número de los distintos depósitos de uranio en un área dada es una variable aleatoria de Poisson con parámetro $\lambda=10$. Si en un periodo de tiempo fijado, cada depósito es descubierto independientemente con probabilidad $\tfrac{1}{50}$, calcular la probabilidad que sea
		
	\begin{parts}
		\part exactamente $1$.
		\part Al menos $1$
		\part A lo más $1$ depósito son descubiertos durante aquel tiempo.
	\end{parts}
	
	\begin{solutionorgrid}
		A
	\end{solutionorgrid}

		\question
	El número de errores de un disco duro es modelado como una variable aleatoria de Poisson con la esperanza de un error en cada Mb, esto es, en cada $2^{20}$ bytes.
	\begin{parts}
	\part ¿Cuál es la probabilidad de que haya al menos un error en un sector de \SI{512}{\byte}?
	\part El disco duro es una unidad de disco de \SI{18.62}{\giga\byte} con \num{39054015} sectores. ¿Cuál es la probabilidad de que haya al menos un error en el disco duro?
	\end{parts}

	\begin{solutionordottedlines}
		\begin{parts}
			\part Modelando la cantidad de errores en los bytes del disco duro como un proceso de Poisson, con intensidad $\lambda$\SI{}{\per\byte}. Además, se tiene la siguiente relación
			\begin{equation*}
				\left(\frac{\lambda}{\cancel{\si{\byte}}}\right)\cdot\left(2^{20}\si[per-mode=fraction]{\cancel\byte}\right)=1,
			\end{equation*}
			por lo que $\lambda=2^{-20}$. Luego, la esperanza de que haya al menos un error en un sector de $\SI{512}{\byte}=2^{9}$ bytes es $\lambda\cdot2^{9}=2^{-20}\cdot2^{9}=2^{-11}=\num{0.00048828125}$.
			\part Sea $Y$ el número de errores en el disco duro. Entonces $Y$ es una variable aleatoria con distribución de Poisson con parámetro
			\begin{equation*}
				\mu=\num{39054015}\times\lambda=\num{39054015}=\num{19069.34326}.
			\end{equation*}
			Luego, la probabilidad pedida es
			\begin{align*}
			\mathds{P}\left(Y\ge1\right)
			&\stackrel{\text{def}}{=}1-\mathds{P}\left(Y=0\right)\\
			&=1-\frac{\mu^{0}e^{-\mu}}{0!}\\
			&=1-\cancelto{0}{e^{-\num{19069.34326}}}\\
			&\approx 1.
			\end{align*}
		\end{parts}
	\end{solutionordottedlines}

%esto es \SI{100}{\mega\byte\per\second},

			\question
		Si $X$ es una variable aleatoria geométrica, mostrar analíticamente que
		\begin{equation*}
			\mathds{P}[X=n+k\mid X>n]=\mathds{P}[X=k]
		\end{equation*}
		Dar un argumento verbal usando la interpretación de una variable aleatoria geométrica en cuanto a porqué la ecuación anterior es verdadera.
	\begin{solutionorbox}
		A
	\end{solutionorbox}

		\question

	Algunas veces el modelo de Poisson es empleada para estudiar el flujo del tránsito. Si el tránsito puede fluir libremente, entonces se comporta como un proceso de Poisson. Un intervalo de tiempo de \SI{20}{\minute} se divide en intervalos de tiempo de \SI{10}{\second}. En un cierto punto a lo largo de la autopista, el número de autos que pasan es registrado para cada intervalo de tiempo de \SI{10}{\second}. Sea $n_j$ el número de intervalos en las que $j$ autos han pasado para $j=0,\ldots,9$. Suponga que uno encuentra

	\centering
	\begin{tabular}{c|cccccccccc}
		$j$ 	& $0$ & $1$ & $2$ & $3$ & $4$ & $5$ & $6$ & $7$ & $8$ & $9$ \\
		\hline
		$n_{j}	$& $19$ & $38$ & $28$ & $20$ & $7$ & $3$ & $4$ & $0$ & $0$ & $1$
	\end{tabular}\quad.

	\justifying
	\noindent Note que el número total de autos que pasan en esos $20$ minutos es $230$.
	\begin{parts}
		\part ¿Qué elegirías para el parámetro de intensidad $\lambda$?\label{12.7a}
		\part Suponga que estima la probabilidad de que cero autos pasan un intervalo de tiempo de \SI{10}{\second} por $n_0$ dividido por el número total de intervalos de tiempo. ¿Eso está de acuerdo (razonablemente) con el valor que sigue desde su respuesta en~\ref{12.7a}?\label{12.7b}
		\part ¿Qué tomarías por la probabilidad de que $10$ autos pasan en un intervalo de tiempo de \SI{10}{\second}?
	\end{parts}

	\begin{solutionordottedlines}
		El número de intervalos de tiempo de \SI{10}{\second} durante \SI{20}{\minute} es $\sum_{j=0}^{9}n_{j}=\num{120}$.
		\begin{parts}
			\part El valor del parámetro $\lambda$ de un proceso de Poisson es obtenido por el siguiente cociente
				\[
					\frac{\text{número de autos}}{\text{tiempo total en segundos}}
					=\frac{\left(19+38+28+20+7+3+4+0+0+1\right)}{\left(\SI{20}{\cancel\minute}\cdot\frac{\SI{60}{\second}}{\SI{1}{\cancel\minute}}\right)}=\frac{23}{120}.\quad
				\]
			\part El parámetro de Poisson con la nueva condición de la parte~\ref{12.7b} es
				\[
					\mathds{P}\left(N=10\right)=\frac{n_{0}}{\sum_{j=0}^{9}n_{j}}=\frac{19}{120},
				\]
			lo cual notamos que el valor es próximo a $\lambda$. Por lo tanto, es una aproximación razonable. Por otro lado, si $\lambda=\tfrac{23}{120}$, la probabilidad de que ningún auto pase en un intervalo de \SI{10}{\second}, esto es, $\mathds{P}\left(N\left(10\right)=0\right)$ es igual a
				\[
					\frac{{\left(\lambda\cdot\SI{10}{\second}\right)}^{0}e^{\left(-\lambda\cdot\SI{10}{\second}\right)}}{0!}
					=\frac{{\left(\frac{23}{120}\cdot\SI{10}{\second}\right)}^{0}e^{\left(-\frac{23}{120}\cdot\SI{10}{\second}\right)}}{0!}
					=\exp\left(-\frac{23}{12}\right)\approx\num{0.147096467}.
				\]
			\part La probabilidad de que diez autos pasan en un intervalo de \SI{10}{\second} es igual a
				\[
					\frac{{\left(\lambda\cdot\SI{10}{\second}\right)}^{10}e^{\left(-\lambda\cdot\SI{10}{\second}\right)}}{10!}
					=\frac{{\left(\frac{23}{120}\cdot\SI{10}{\second}\right)}^{10}e^{\left(-\frac{23}{120}\cdot\SI{10}{\second}\right)}}{10!}
					\approx\num{2.712096323e-5}.
				\]
		\end{parts}
	\end{solutionordottedlines}

		\question
	Sea $X$ una variable aleatoria de Poisson con parámetro $\mu$.
	\begin{parts}
		\part Calcule $\mathds{E}\left[X\left(X-1\right)\right]$.\label{12.8}
		\part Calcule $\mathrm{Var}\left(X\right)$, usando el siguiente hecho:
		\begin{equation*}
		\mathrm{Var}\left(X\right)=\mathds{E}\left[X\left(X-1\right)\right]+\mathds{E}\left[X\right]-{\left(\mathds{E}\left[X\right]\right)}^2.
		\end{equation*}
	\end{parts}

	\begin{solutionordottedlines}
		\begin{parts}
			\part Sea $X$ una variable aleatoria de Poisson con parámetro $\mu$. Entonces
			\begin{align*}
				\mathds{E}\left[X\left(X=1\right)\right]
				&\stackrel{\text{def}}{=}\sum_{k=0}^{\infty}\frac{k\left(k-1\right)e^{-\mu}\mu^{k}}{k!}\\
				&=\sum_{k=0}^{\infty}\frac{\cancel{k}\cancel{\left(k-1\right)}e^{-\mu}\mu^{k}}{\cancel{k}\cdot\cancel{\left(k-1\right)}\cdot\left(k-2\right)!},\\
				\intertext{ahora expandimos los dos primeros términos de la serie}
				\mathds{E}\left[X\left(X=1\right)\right]
				&=\cancelto{0}{\frac{0\cdot\left(-1\right)\cdot e^{-\mu}\mu^{0}}{0!}}+\cancelto{0}{\frac{1\cdot\left(0\right)\cdot e^{-\mu}\mu^{1}}{1!}}+\sum_{k=2}^{\infty}\frac{e^{-\mu}\mu^{k}}{\left(k-2\right)!}\\
				&=\mu^{2}\sum_{k=2}^{\infty}\frac{e^{-\mu}\mu^{k-2}}{\left(k-2\right)!}.\\
				\intertext{Ahora, cambiaremos adecuadamente el índice $k$ por $j$}
				\mathds{E}\left[X\left(X=1\right)\right]
				&=\mu^{2}\sum_{j=0}^{\infty}\frac{e^{-\mu}\mu^{j}}{\left(j\right)!}\\
				&=\mu^{2}\cdot e^{-\mu}\left(\sum_{j=0}^{\infty}\frac{\mu^{j}}{\left(j\right)!}\right),\\
				\intertext{pero, el valor entre paréntesis es $e^{\mu}$}
				\mathds{E}\left[X\left(X=1\right)\right]
				&=\mu^{2}\cdot \cancelto{1}{\left(e^{-\mu}\cdot e^{\mu}\right)}=\mu^{2}.
			\end{align*}
			\part Ya que $\mathrm{Var}\left(X\right)=\mathds{E}\left[X\left(X-1\right)\right]+\mathds{E}\left[X\right]-{\left(\mathds{E}\left[X\right]\right)}^2$, reemplazamos la expresión obtenido en la parte~\ref{12.8}
			\begin{equation*}
			\mathrm{Var}\left(X\right)=\cancel{\mu^{2}}+\mu-\cancel{{\left(\mu\right)}^{2}}=\mu.
			\end{equation*}
		\end{parts}
	\end{solutionordottedlines}

		\question
	 Sea $Y_1$ y $Y_2$ dos variables aleatorias de Poisson independientes con parámetros $\mu_1$ y $\mu_2$, respectivamente. Muestre que $Y=Y_1+Y_2$ también es una distribución de Poisson. En vez de usar la regla de adicción en la Sección 11.1 como en el ejercicio 11.2, puede probar esto sin ningún cálculo considerando el número de puntos de un proceso de Poisson (con intensidad $1$) en dos intervalos disjuntos de longitud $\mu_1$ y $\mu_2$.
	
	\begin{solutionordottedlines}
		En un proceso de Poisson con intensidad $1$, el número de puntos en un intervalo de longitud $t$ tiene una distribución de Poisson con parámetro $t$. Así el número $Y_{1}$ de puntos en $\left(0,\mu_1\right)$ es una variable aleatoria con distribución $\mathrm{Pois}\left(\mu_1\right)$ y $Y_{2}$ en el intervalo $\left[\mu_1+\mu_1+\mu_2\right]$ es una variable aleatoria con distribución $\mathrm{Pois}\left(\mu_2\right)$. Pero la suma de $Y_{1}+Y_{2}$ de estos son igual al número de puntos en el intervalo $\left[0,\mu_1+\mu_2\right]$, por lo que se concluye que la variable aleatoria $Y_{1}+Y_{2}$ es otra variable aleatoria con distribución $\mathrm{Pois}\left(\mu_1+\mu_2\right)$.
	\end{solutionordottedlines}

		\question
	Sea $X$ una variable aleatorias con una distribución $\mathrm{Pois}\left(\mu\right)$. Muestre lo siguiente. Si $\mu<1$, entonces las probabilidades $\mathds{P}\left(X=k\right)$ son primero crecientes, luego decreciente (confer Figura~\ref{fig:12_1}). ¿Qué ocurre si $\mu=1$?
	
	\begin{figure}[ht!]
		\centering
		\includegraphics[width=0.70\paperwidth]{img12_1}
		\caption{Las funciones de probabilidad de masa de las distribuciones de $\mathrm{Pois}\left(0.9\right)$ y $\mathrm{Pois}\left(5\right)$.}
		\label{fig:12_1}
	\end{figure}

	\begin{solutionordottedlines}
		Definimos las probabilidades
		\begin{equation*}
		p_{k}\stackrel{\text{def}}{=}\mathds{P}\left(X=k\right)=\frac{\mu^{k}e^{-\mu}}{k!}\quad\text{para } k=0,1,\ldots
		\end{equation*}
		Ahora, calculemos el siguiente cociente
		\begin{equation*}
		\frac{p_{k+1}}{p_{k}}
		=\frac{\mathds{P}\left(X=k+1\right)}{\mathds{P}\left(X=k\right)}
		=\frac{\frac{\mu^{k+1}e^{-\mu}}{\left(k+1\right)!}}{\frac{\mu^{k}e^{-\mu}}{k!}}
		=\frac{\frac{\cancel{\mu^{k}}\cdot\mu\cdot \cancel{e^{-\mu}}}{\left(k+1\right)\cdot\cancel{ k!}}}{\frac{\cancel{\mu^{k}}\cdot\cancel{e^{-\mu}}}{\cancel{k!}}}
		=\frac{\mu}{k+1}.
		\end{equation*}
		Además, si $p_{k}<p_{k+1}$ o $p_{k}>p_{k+1}$, entonces $\frac{p_{k+1}}{p_{k}}>1$ o $\frac{p_{k+1}}{p_{k}}<1$. De esto se sigue que
		\begin{itemize}
			\item Para $\mu<1$, la probabilidad $\mathds{P}\left(X=k\right)$ es decreciente, y
			\item para $\mu>1$, la probabilidad es creciente siempre que $\tfrac{\mu}{k+1}>1$ y decreciente desde el momento en el que esta fracción se vuelve menor que uno.
		\end{itemize}
		Por último, si $\mu={1}$, entonces $p_{0}=p_{1}>p_{2}>p_{3}>\cdots$.
	\end{solutionordottedlines}

		\question

	Considere el proceso de Poisson unidimensional con intensidad $\lambda$. Muestre que el número de puntos en $\left[0,t\right]$, \emph{dado} que el número de puntos en $\left[0,2t\right]$ es igual a $n$, tiene una distribución $\mathrm{Bin}\left(n,\tfrac{1}{2}\right)$.
 
	\noindent\emph{Ayuda:} escriba el evento $\left\{N\left(\left[0,s\right]\right)=k,N\left(\left[0,2s\right]\right)=n\right\}$ como la intersección de eventos (¡independientes!) $\left\{N\left(\left[0,s\right]\right)=k\right\}$ y $\left\{N\left(\left[0,2s\right]\right)=n-k\right\}$.

	\begin{solutionordottedlines}
		De la ayuda, obtenemos
		\begin{align*}
			\mathds{P}\left(N\left(\left[0,s\right]\right)=k,N\left(\left[0,2s\right]\right)=n\right)
			&=\mathds{P}\left(N\left(\left[0,s\right]\right)=k,N\left(\left[s,2s\right]\right)=n-k\right)
		\end{align*}
	\end{solutionordottedlines}

		\question

	Consideremos el proceso de Poisson unidimensional. Suponga que para algunos $\alpha>0$ se da que hay exactamente dos puntos $\left[0,a\right]$, o en otras palabras: $N_\alpha=2$. El objetivo de este ejercicio es determinar la distribución conjunta de $X_1$ y $X_2$, las ubicaciones de los dos puntos, sujeto a $N_\alpha=2$.

	\begin{parts}
		\part Pruebe que para $0<s<t<\alpha$\label{12.12a}
			\[
				\mathds{P}\left(X_1\le X_2\le t, N_\alpha=2\right)=\mathds{P}\left(X_2\le t, N_\alpha=2\right)-\mathds{P}\left(X_1>s,X_2\le t, N_\alpha=2\right).
			\]
		\part Deduzca de~\ref{12.12a} que\label{12.12b}
			\[
				\mathds{P}\left(X_1\le s,X_2\le t,N_\alpha=2\right)=e^{-\lambda\alpha}\left(\frac{\lambda^2t^2}{2!}-\frac{\lambda^2{\left(t-s\right)}^2}{2!}\right).
			\]
		\part Deduzca de~\ref{12.12b} que $0<s<t<\alpha$
			\[
				\mathds{P}\left(X_1\le s, X_2\le t\mid N_\alpha=2\right)=\frac{t^2-{\left(t-s\right)}^2}{\alpha^2}.
			\]
	\end{parts}

	\begin{solutionordottedlines}
		
	\end{solutionordottedlines}

		\question

	Al caminar por un prado encontramos dos tipos de flores, margaritas y dientes de león. Mientras caminamos en línea recta, modelamos las posiciones de las flores que encontramos con un proceso de Poisson con intensidad $\lambda$. Parece que aproximadamente una de cada cuatro flores es margarita. Olvidando los dientes de león, ¿cómo se ve el proceso de las \emph{margaritas}? Esta pregunta será respondida con los siguientes pasos:
	\begin{parts}
		\part Sea $N_t$ el número total de flores, $X_t$ el número de margaritas, e $Y_t$ el número de dientes de león que encontramos durante los primeros $t$ minutos de nuestro paseo. Note que $X_t+Y_t=N_t$. Suponga que cada flor es un margarita con probabilidad $1/4$, independiente de las otras flores. Argumenta que\label{12.13a}
			\[
				\mathds{P}\left(X_t=n,Y_t=m\mid N_t=n+m\right)=\binom{n+m}{n}{\left(\frac{1}{4}\right)}^n{\left(\frac{3}{4}\right)}^m.
			\]
		\part Muestre que
			\[
				\mathds{P}\left(X_t=n,Y_t=m\right)=\frac{1}{n!}\frac{1}{m!}{\left(\frac{1}{4}\right)}^n{\left(\frac{3}{4}\right)}^me^{-\lambda t}{\lambda t}^{n+m},
			\]
		condicionando sobre $N_t$ y usando~\ref{12.13a}.
		\part Escribiendo $e^{-\lambda t}=e^{-\left(\lambda/4\right)t}e^{-\left(3\lambda/4\right)t}$ y sumando sobre $m$, muestre que\label{12.13c}
			\[
				\mathds{P}\left(X_t=n\right)=\frac{1}{n!}e^{-\left(\lambda/4\right)t}{\left(\frac{\lambda t}{4}\right)}^n.
			\]
		Ya que está claro que los números de margaritas que encontramos en intervalos de tiempo disjuntos son independientes, podríamos concluir de~\ref{12.13c} que el proceso $\left(X_t\right)$ es de nuevo un proceso de Poisson, con intensidad $\lambda/4$. Uno dice a menudo que el proceso $\left(X_t\right)$ es obtenido por \emph{adelgazamiento} del proceso $\left(N_t\right)$. En nuestro ejemplo este corresponde a recoger todos los dientes de león.
	\end{parts}

	\begin{solutionordottedlines}
		
	\end{solutionordottedlines}

		\question
	En este ejercicio nos fijamos en un ejemplo simple de variables aleatorias $X_n$ que tiene la propiedad que sus distribuciones convergen a las distribuciones de una variable aleatorias $X$ cuando $n\to\infty$. Sea para $n=1,2,\ldots$ las variables aleatorias definidas por
	\begin{equation*}
	P\left(X_n=0\right)=1-\frac{1}{n}\quad\text{y}\quad\mathds{P}\left(X_n=7n\right)=\frac{1}{n}.
	\end{equation*}
	\begin{parts}
		\part Sea $X$ una variable aleatorias que es igual a $0$ con probabilidad $1$. Muestre que para todo $a$, las funciones de masa de probabilidad $p_{X_m}(a)$ de $X_n$ converge a la función de probabilidad de masa $p_X(a)$ de $X$ cuando $n\to\infty$. Note que $\mathds{E}\left[X\right]=0$.
		\part Muestre que sin embargo $\mathds{E}\left[X_n\right]=7$ para todo $n$.
	\end{parts}

	\begin{solutionordottedlines}
		
	\end{solutionordottedlines}

	\end{questions}

	%% --------------------
%% |   Bibliography   |
%% --------------------
%% Add entry to the table of contents for the bibliography

\vfill                                                                           
\nocite{*}                                                                       
\printbibliography[title={Referencias bibliográficas},heading=bibintoc]

	\section*{Puntuación}
	
	\begin{center}
		\settabletotalpoints{20}
		\gradetable[v][questions]
	\end{center}

\end{document}