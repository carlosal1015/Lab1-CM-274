\question\label{q:1}%[10]
	Cinco números distintos se distribuyen aleatoriamente entre los jugadores del $1$ al $5$. Cuando dos jugadores comparan sus números, el que tiene el más alto es declarado ganador. Inicialmente, los jugadores $1$ y $2$ comparan sus números; el ganador se compara con el jugador $3$, y así sucesivamente. Sea $X$ el número de veces que el jugador $1$ es un ganador. Encuentra $\mathds{P}[X = i]$, $i = 0, 1, 2, 3, 4$.

	\ifprintanswers
	\begin{figure}[!ht]
		\centering
		\includegraphics[width=.5\textwidth]{fig1.pdf}
		\caption{Diagrama de árbol de las rondas del juego de la pregunta~\ref{q:1}.}
		\label{fig:1}
	\end{figure}
	\fi
	
	\begin{solutionorbox}
		Una variable aleatoria $X$ que asume los valores $1,2,3,4,5$ es llamado una variable aleatoria discreta.

		Sea $X$ una variable aleatoria cuyos valores de $x$ están en $\left\{1,2,3,4,5\right\}$. Ahora
		\begin{align*}
			f(1) &= \mathds{P}\left(X=1\right)=\frac{}{}, & 	f(2) &= \mathds{P}\left(X=2\right)=\frac{}{}, \\
			f(3) &= \mathds{P}\left(X=3\right)=\frac{}{}, & 	f(4) &= \mathds{P}\left(X=4\right)=\frac{}{}, \\
			f(5) &= \mathds{P}\left(X=5\right)=\frac{}{}, & 	&
		\end{align*}
		Por lo tanto, la distribución de probabilidad de $X$ es	
		\centering
		\begin{tabular}{c|ccccc}
			$x$ 	& 1 & 2 & 3 & 4 & 5 \\
			\hline
			$f(x)$& 1 & 2 & 3 & 4 & 5
		\end{tabular}
	\end{solutionorbox}