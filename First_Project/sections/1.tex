\question\label{q:1}%[10]
	Cinco números distintos se distribuyen aleatoriamente entre los jugadores del $1$ al $5$. Cuando dos jugadores comparan sus números, el que tiene el más alto es declarado ganador. Inicialmente, los jugadores $1$ y $2$ comparan sus números; el ganador se compara con el jugador $3$, y así sucesivamente. Sea $X$ el número de veces que el jugador $1$ es un ganador. Encuentra $\mathds{P}[X = i]$, $i = 0, 1, 2, 3, 4$.

	\ifprintanswers
	\begin{figure}[!ht]
		\centering
		\includegraphics[width=.5\textwidth]{fig1.pdf}
		\caption{Diagrama de árbol de las rondas del juego de la pregunta~\ref{q:1}.}
		\label{fig:1}
	\end{figure}
	\fi
	
	\begin{solutionorbox}
		Sea $X$ una variable aleatoria discreta cuyos valores de $x$ están en $\left\{0,1,2,3,4\right\}$ que representa el número de rondas que gana el Jugador $1$ durante el torneo. Se asumirá que los números son asignados a los jugadores al azar, con cada permutación de los números con la misma probabilidad.

		El jugador 1 es (a priori) igualmente probable que obtenga cualquiera de los cinco números.
		\begin{itemize}
			\item Si obtienen un $1$, están seguros de ganar $0$ rondas.
			\item Si obtienen un $2$, ganan $0$ rondas o $1$ ronda. ¿Cuál es la única circunstancia bajo la cual ganan $1$ vuelta?
			\item Si obtienen un $3$, hay dos números más altos ($A$, ya que no necesitamos distinguir entre $4$ y $5$) y dos números más bajos (llamémoslos $B$). ¿Cuántas permutaciones diferentes hay de $AABB$? Una pareja son $ABAB$ y $BABA$. ¿En qué casos el Jugador 1 gana $0$, $1$ o $2$ rondas? Por ejemplo, $AABB$ lleva a ganar $0$ rondas.
			\item Si obtienen un $4$, ganan de $0$ a $3$ rondas. Depende solo de donde se encuentre el $5$.
			\item Si obtienen un $5$, están seguros de ganar las $4$ rondas.
		\end{itemize}

		Sea $E_n$ el evento en el que el jugador $1$ tenga el número más alto entre los jugadores desde el $1$ hasta $n$. Entonces $\mathds{P}\left(E_{n}\right)=\tfrac{1}{n}$. Además $E_{n+1}\subseteq E_{n}$, y por lo tanto
		\begin{equation*}
			\mathds{P}\left(E_{n}\setminus E_{n+1}\right)=\mathds{P}\left(E_{n}\right)-\mathds{P}\left(E_{n+1}\right)=\frac{1}{n}-\frac{1}{n+1}=\frac{1}{n\left(n+1\right)}.
		\end{equation*}
		
		Como $E_{n}\setminus E_{n+1}$ es el evento $X=n-1$, tenemos $\mathds{P}\left(X=n\right)=\tfrac{1}{\left(n+1\right)\left(n+2\right)}$. Esto se mantiene para $0\le n< 4$. Para $n=4$, no hay más rondas para ganar, por lo que la probabilidad de ganar $4$ rondas es solo $\mathds{P}\left(E_{5}\right)=\tfrac{1}{5}$. Ahora, las probabilidades develadas son 
		\begin{align*}
			f(X=0) &= \mathds{P}\left(X=0\right)=\frac{1}{2},	& 	f(X=1)	&= \mathds{P}\left(X=1\right)=\frac{1}{6}, \\
			f(X=2) &= \mathds{P}\left(X=2\right)=\frac{1}{12},& 	f(X=3)	&= \mathds{P}\left(X=3\right)=\frac{1}{20}, \\
			f(X=4) &= \mathds{P}\left(X=4\right)=\frac{1}{5}.	&						&
		\end{align*}
		Por lo tanto, la distribución de probabilidad de $X$ es

		\centering
		\begin{tabular}{c|ccccc}
			$x$ 	& $0$ & $1$ & $2$ & $3$ & $4$ \\
			\hline
			$f(x)$& $\frac{1}{2}$ & $\frac{1}{6}$ & $\frac{1}{12}$ & $\frac{1}{20}$ & $\frac{1}{5}$
		\end{tabular}\quad.
	\end{solutionorbox}