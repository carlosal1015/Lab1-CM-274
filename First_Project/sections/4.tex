	\question%[10]
		Supuesto que el número de ventas que ocurren en un tiempo específico es una variable aleatoria de Poisson con parámetro $\lambda$. Si cada evento es contado con probabilidad $p$, independiente un evento de otro evento, mostrar que el numero de eventos que son contados es una V.A de Poisson con parámetro $\lambda p$. También, dar un argumento intuitivo de porqué esto debería ser así. Como una aplicación del párrafo anterior, suponer que el número de los distintos depósitos de uranio en un área dada es una variable aleatoria de Poisson con parámetro $\lambda=10$. Si en un periodo de tiempo fijado, cada depósito es descubierto independientemente con probabilidad $\tfrac{1}{50}$, calcular la probabilidad que sea
		
	\begin{parts}
		\part exactamente $1$.
		\part Al menos $1$
		\part A lo más $1$ depósito son descubiertos durante aquel tiempo.
	\end{parts}
	
	\begin{solutionorgrid}
		A
	\end{solutionorgrid}